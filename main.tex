\documentclass[10pt]{book}

\usepackage[heading]{ctex}
\usepackage[paper=b5paper]{geometry}
\usepackage{amsmath}
\usepackage{amssymb}
\usepackage{amsthm}
% \usepackage{newtxtext}
% \usepackage{newtxmath}
\usepackage{mathrsfs}
\usepackage[backend=biber,style=caspervector,utf8,seconds=true]{biblatex}
% \usepackage[backend=biber,style=gb7714-2015]{biblatex}
\usepackage[shortlabels]{enumitem}
\setlist{nosep}
\usepackage{tikz-cd}
\usepackage[colorlinks]{hyperref}

\addbibresource{biblio.bib}
\defbibheading{bibliography}[参考文献]{\subsection*{#1}}

% Theorem environments
\theoremstyle{definition}
\newtheorem{defn}{定义}[chapter]
\newtheorem{sym}[defn]{记号}
\newtheorem{eg}{例}[chapter]
\theoremstyle{plain}
\newtheorem{thm}[defn]{定理}
\newtheorem{lem}[defn]{引理}
\newtheorem{col}[defn]{推论}
\newtheorem{prop}[defn]{命题}
\newtheorem*{pro}{问题}
\theoremstyle{remark}
\newtheorem{rem}[defn]{评注}
\newcounter{exercise}
\newtheorem{ex}{习题}[chapter]


\DeclareMathOperator{\Aut}{Aut}
\DeclareMathOperator{\Ann}{Ann}
\DeclareMathOperator{\Gal}{Gal}
\DeclareMathOperator{\GF}{GF}
\DeclareMathOperator{\ch}{char}
\DeclareMathOperator{\GL}{GL}
\DeclareMathOperator{\SL}{SL}
\DeclareMathOperator{\Quot}{Quot}
\DeclareMathOperator{\Spec}{Spec}
\DeclareMathOperator{\MaxSpec}{MaxSpec}

\title{一份速通性质的近世代数讲义}
\author{魔法少女Alkali}
\date{最后编译: \today}

\begin{document}
\maketitle

\thispagestyle{empty}

\frontmatter
\tableofcontents
\chapter{前言}

本讲义是2023年7月作者举办的面向新二年级同学的近世代数讲义.
使用讲义的时候, 作者默认了读者学习了北师大的高等代数I, II课程.

部分参考书籍与我们推荐阅读的书籍如下:
对于中文书籍, \parencite{BNU}~是北师大近世代数课程的教材; 而~\parencite{wwli}~则是一本比较``升级''的教材, 介绍了更加现代的内容.
对于英文书籍, \parencite{Hungerford}~是标准的教材;
\parencite{Chapter0}~是一本偏向入门的书籍, 但在一开始便以较高观点引入范畴等内容, 适合作为研究生级别教材学习;
\parencite{Lang}~是著名的字典, 以大而全闻名, 适合用来查阅.

\chapter{预备知识}

我们具体列举希望读者掌握的预备知识如下.

首先, 讲义中会使用与北师大高等代数课程不同的记号$\mathbb{Z}/n\mathbb{Z}$表示模$n$剩余类环, 并且会直接使用同余记号$\mathrm{mod}$记属于同一等价类的元素.

接下来我们定义群环域.
\begin{defn}设$X$是一个集合, 具有二元运算$*:X\times X\to X$, 并有公理
    \begin{itemize}
        \item[G1] 对$a,b,c\in X$有$(a*b)*c=a*(b*c)$;
        \item[G2] 存在$e\in X$, 使得对任意$a\in X$有$a*e=e*a=a$;
        \item[G3] 对$a\in X$, 存在$b\in X$使得$a*b=b*a=e$;
        \item[Ab] 对任意$a,b\in X$, 有$a*b=b*a$;
    \end{itemize}
    如果$X$上还有另一二元运算$\cdot:X\times X\to X$, 此时还有公理
    \begin{itemize}
        \item[R1] 对$a,b,c\in X$有$(a\cdot b)\cdot c=a\cdot(b\cdot c)$;
        \item[R2] 存在$1\in X$, 使得对任意$a\in X$有$a\cdot 1=1\cdot a=a$;
        \item[R3] 对任意$a,b\in X$, 有$a\cdot b=b\cdot a$;
        \item[Ds] 对$a,b,c\in X$有$(a*b)\cdot c=a\cdot c*b\cdot c, a\cdot(b*c)=a\cdot b*a\cdot c$;
        \item[F] 对任意$a\in X\backslash\{e\}$, 存在$b\in X$使得$a\cdot b=1$;
    \end{itemize}
    如果$X$满足G1$\sim$G3, 那么称$X$是一个\textbf{群}; 如果群$X$还满足Ab, 则称$X$是一个\textbf{Abel群}.
    如果$X$是Abel群, 且满足R1与Ds, 那么称$X$是一个\textbf{环}; 如果环$X$满足R2, 那么称$X$\textbf{含幺}; 如果环$X$满足R3, 那么称$X$是\textbf{交换环}.
    如果$X$是交换环且满足F, 那么称$X$是一个\textbf{域}.
\end{defn}

\begin{sym}
    习惯上, 一般对群的运算会采用两种记号: 一种是乘法记号$\cdot$, 在实际书写中会直接省略这个点; 另一种是加法记号$+$.
    乘法记号会用在一般的群或者环满足R1与Ds的运算上, 加法记号会用在Abel群的运算上.
    运用乘法记号时, G3中定义的逆元会记作$a^{-1}$.
    运用加法记号时, G2中定义的加法零元记为$0$, G3中定义的逆元记为$-a$.
    对域而言, F中定义的逆元记为$a^{-1}$.

    本讲义中如果不另外说明, 环都是含幺的.

    习惯上会用一些特定的字母表示特定的代数结构, 例如群用$G$表示, 环用$R$表示, 交换环用$A$表示, 域用$F$或$k$表示.
\end{sym}

然后是置换群的基本概念.
\begin{prop}
    集合$\{1,2,\cdots,n\}$到自身的双射构成群, 记为$S_n$, 称为{\bf $n$阶置换群}.
\end{prop}

\begin{sym}
    对$\sigma\in S_n$, 我们会用
    \[\begin{pmatrix}
        1 & \cdots & n\\
        \sigma(1) & \cdots & \sigma(n)
    \end{pmatrix}\]
    来表示一个置换.
\end{sym}

\begin{prop}
    一个{\bf 轮换}定义为
    \[\begin{pmatrix}
        a_1 & a_2 & \cdots & a_n\\
        a_2 & a_3 & \cdots & a_1
    \end{pmatrix}\]
    记为$(a_1a_2\cdots a_n)$.
    每一个置换都可以写成不相交轮换的乘积, 这种写法在不计次序的意义下唯一.
\end{prop}

最后是关于矩阵的两种群.
\begin{defn}
    域$k$上的$n$阶\textbf{一般线性群}$\GL_n(k)$定义为$k$上所有可逆的$n$阶矩阵构成的群.
    域$k$上的$n$阶\textbf{特殊线性群}$\SL_n(k)$定义为$k$上所有行列式为$1$的$n$阶矩阵构成的群.
\end{defn}

\mainmatter
\chapter{群与环的结构}\label{structures}
\section{同态}
代数学研究代数对象及它们之间的态射.
这里的``态射''指的是保持代数运算结构的映射, 一般称为\textit{同态}.
严格的定义如下:

\begin{defn}
    \begin{itemize}
        \item 设$G,G'$是两个群, 映射$f:G\to G'$称为 ($G$到$G'$的) 一个\textbf{群同态}, 如果$f$满足对任意$a,b\in G$有$f(ab)=f(a)f(b)$.
        \item 设$R,R'$是两个环, 映射$f:R\to R'$称为 ($R$到$R'$的) 一个\textbf{环同态}, 如果$f$满足
        \begin{enumerate}[(1)]
            \item $f(a+b)=f(a)+f(b)$;
            \item $f(ab)=f(a)f(b)$;
            \item $f(1_R)=1_{R'}$.
        \end{enumerate}
        其中$1_R,1_{R'}$分别是$R$与$R'$的乘法幺元.
    \end{itemize}
\end{defn}

在本讲义中, 我们会直接使用同态的相关运算性质而不加证明, 读者有疑问时不妨自行证明, 大部分的证明都与线性映射类似\footnote{实际上, 线性映射就是向量空间之间的同态.}.

使用同态, 我们可以定义子群及子环:

\begin{defn}\label{subring}
    设$X,X'$是群 (环), $X'\subset X$, 如果包含映射$i:X'\to X$是群 (环) 同态, 那么称$X'$是$X$的子群 (环).
    如果$X'\neq\{e\}(\{0\}),X$, 那么称$X'$是真子群 (环).
\end{defn}

定义~\ref{subring}~无外乎就是说$X'$在$X$的运算下成群或者环, 请读者自行证明这一点.
等价的一些检验方法有
\begin{itemize}
    \item (对群) 关于除法封闭;
    \item (对环) 关于减法和乘法封闭, 包含幺元.
\end{itemize}
等价性在高等代数I课程中有过证明.

对于两个群或者环, 我们可以定义他们之间的\textit{同构}, 这时它们在代数运算的意义下可以看作是一样的.
\begin{defn}
    设$X, X'$是两个群或者环, 如果存在同态$f:X\to X',\ g:X'\to X$使得$f\circ g=\mathrm{id}_{X'},\ g\circ f=\mathrm{id}_X$, 那么称$X$与$X'$同构.
\end{defn}

\begin{defn}
    一个群$G$的所有自同构构成一个群, 称为$G$的自同构群, 记为$\Aut(G)$.
\end{defn}

\begin{ex}\label{inverse}
    如果$f:X\to X'$是同态且是双射, 证明$f$是同构.
\end{ex}

对同态, 我们会考虑同态的核.

\begin{defn}
    对群而言, 设$f:G\to G'$是一个群同态, 定义$f$的\textbf{核}$\ker f:=f^{-1}(e)$.
    对环而言, 设$g:R\to R'$是一个环同态, 定义$g$的核为$\ker g:=g^{-1}(0)$.
\end{defn}

核的定义与线性映射的核的定义是相同的.
同态的核是很重要的研究对象, 我们将要用核定义两种很重要的子集.

\begin{lem}\label{ker is subgroup}
    如果$f:G\to G'$是同态, 那么$\ker f$是$G$的子群.
\end{lem}
\begin{proof}
    我们证明$\ker f$在$G$的运算下成群.
    这只需要证明$\ker f$关于除法封闭.
    对$a,b\in\ker f$, 考虑$f(ab^{-1})$.
    由于$f(b)f(b^{-1})=f(bb^{-1})=f(e)$, 而$f(e)=f(ee)=f(e)f(e)$得出$f(e)=e$, 所以$f(b^{-1})=(f(b))^{-1}$.
    因此$f(ab^{-1})=f(a)(f(b))^{-1}=ee^{-1}=e$, 即$ab^{-1}\in\ker f$.
\end{proof}

\begin{defn}
    设$G$是一个群, 如果$N\subset G$是一个同态的核, 那么称$N$是$G$的一个\textbf{正规子群}, 并记作$N\lhd G$.
    设$R$是一个环, 如果$\mathfrak{a}\subset R$是一个同态的核, 那么称$\mathfrak{a}$是$R$的一个\textbf{理想}.

    同样的, 当$N\neq\{e\},G$, $\mathfrak{a}\neq\{0\},R$时, 称$N$或$\mathfrak{a}$为真正规子群或真理想.
\end{defn}

\begin{prop}[正规子群的性质]
    设$N\lhd G$, 那么对任意$g\in G$及$n\in N$, 有$gng^{-1}\in N$.
\end{prop}
\begin{proof}
    设$N=\ker f$, 那么有$f(gng^{-1})=f(g)f(n)(f(g))^{-1}=f(n)=e$, 所以$gng^{-1}\in\ker f=N$.
\end{proof}

\begin{prop}[理想的性质]\label{defn of ideal}
    设$\mathfrak{a}$是$R$的理想.
    \begin{enumerate}[\rm (1)]
        \item $\mathfrak{a}$是$R$的子加群;
        \item 对任意的$a\in\mathfrak{a}$与$r\in R$, 有$ar,ra\in\mathfrak{a}$.
    \end{enumerate}
\end{prop}
\begin{proof}
    (1) 环同态是两个环之间关于加法的群同态.\\
    (2) 设$\mathfrak{a}=\ker f$, 那么有$f(ar)=f(a)f(r)=0\cdot f(r)=0$, 因此$ar\in\mathfrak{a}$.
    同理$ra\in\mathfrak{a}$.
\end{proof}

\begin{rem}
    从以上命题可以看出, 我们定义的理想在一些教材中会被合理地称作``双侧理想'', 除此之外还有所谓的``左理想''和``右理想'', 不过我们目前不讨论这些精细的定义.
\end{rem}

\section{等价关系与商}

我们熟悉如下的一个命题:
\begin{prop}
    集合$X$上的一个等价关系$\sim$唯一决定$X$的一个分划, 这个分划得到的等价类集称为$X$的{\bf 商集}, 记为$X/\sim$.
\end{prop}

我们在本节考虑两种等价关系:
首先是子群诱导的等价关系, 这种等价关系可以得到关于有限群子群阶数 (即元素个数) 的整除关系;
其次是正规子群和理想诱导的等价关系, 这种等价关系可以使得商集上具有良定义的代数运算.

\subsection{Lagrange定理}

设群$G$具有子群$G'$.
考虑关系$a\sim b\iff a^{-1}b\in G'$.
我们验证这是一个等价关系:
\begin{itemize}
    \item[自反性] $a^{-1}a=e\in G'$;
    \item[对称性] 如果$a^{-1}b\in G'$, 那么$b^{-1}a=(ab^{-1})^{-1}\in G'$;
    \item[传递性] 如果$a^{-1}b,b^{-1}c\in G$, 那么$a^{-1}c=a^{-1}bb^{-1}c\in G$.
\end{itemize}
因此$\sim$给出$G$的一个分划.
这个分划具有如下的一个性质:

\begin{prop}\label{cardinal of coset}
    $G/\sim$的每个等价类的基数均相等, 且都等于$|G'|$.
\end{prop}
\begin{proof}
    设$C$是一个等价类, 我们建立$G'$到$C$的一个双射.
    任取$a\in C$, 定义
    \begin{align*}
        \varphi:G'&\to C\\
        g&\mapsto ag
    \end{align*}
    首先这个映射是良定义的, 因为有$a^{-1}ag=g\in G'$.
    其次这个映射一定是单射, 因为$ag=ag'\implies g=g'$.
    最后这个映射一定是满射, 因为对任意$b\in C$, 设$a^{-1}b=g'$, 那么就有$b=aa^{-1}b=ag'$.
    因此$\varphi$是一个双射, 有$|C|=|G'|$.
\end{proof}

通过命题~\ref{cardinal of coset}~的证明, 我们可以看出$G/\sim$的每一个等价类都由$G$中的一个元素左乘$G'$中所有元素得到.
于是我们定义
\begin{defn}
    定义$G'$的一个\textbf{左陪集}为$aG':=\{ag'\in G|\ g'\in G'\}$.
\end{defn}

将等价关系与商集翻译称陪集的语言就是
\begin{prop}
    左陪集$aG'$与$bG'$相等当且仅当$a^{-1}b\in G'$.
\end{prop}
\begin{prop}
    设$G'$是$G$的子群, 那么适当选取代表元, $G$有陪集分解$G=\coprod aG'$.
\end{prop}

当$G$是有限群时, 通过陪集分解, 我们能立刻得到
\begin{thm}[Lagrange]
    设$G$是有限群, $G'$是$G$的子群, 那么$G'$的阶整除$G$的阶.
\end{thm}

与左陪集相同, 我们也可以定义右陪集:
\begin{defn}
    定义$G'$的一个\textbf{右陪集}为$G'a:=\{g'a\in G|\ g'\in G'\}$.
\end{defn}
\begin{prop}
    右陪集$G'a$与$G'b$相等当且仅当$ab^{-1}\in G'$.
\end{prop}
\begin{prop}
    设$G'$是$G$的子群, 那么适当选取代表元, $G$有陪集分解$G=\coprod G'a$.
\end{prop}

\subsection{商群与商环}

首先, 我们证明正规子群的陪集类上将会存在群的乘法.
\begin{thm}
    设群$G$与子群$N\subset G$满足对任意$g\in G$及$n\in N$有$gng^{-1}\in N$, 那么$N$的陪集类构成一个群, 记为$G/N$, 并且$\pi:G\to G/N, a\mapsto aN$构成典范同态.
\end{thm}
\begin{proof}
    我们在$G/N$上定义乘法
    \[(aN,bN)\mapsto abN\]
    一旦证明这个乘法是良定义的, 将立刻得到$\varphi$是同态.
    如果$c\in aN,d\in bN$, 有
    \[(ab)^{-1}cd=b^{-1}a^{-1}cd=b^{-1}(a^{-1}c)b(b^{-1}d)\in N\]
    所以乘法是良定义的. $G/N$显然在这个乘法下成群.
\end{proof}

\begin{col}
    $N\lhd G$当且仅当对任意$g\in G$及$n\in N$, 有$gng^{-1}\in N$.
\end{col}

\begin{ex}
    证明正规子群的左陪集与右陪集相等, 即$N\lhd G$时有$aN=Na$.
\end{ex}

对环而言, 也有类似结论, 证明也是类似的.
\begin{thm}
    设$\mathfrak{a}$是环$R$的子加群, 满足命题~\ref{defn of ideal}~中的性质, 那么$\mathfrak{a}$的陪集类构成一个环, 记为$R/\mathfrak{a}$, 并且$\pi:R\to R/\mathfrak{a},r\to r+\mathfrak{a}$构成典范同态.
\end{thm}

\begin{col}
    命题~\ref{defn of ideal}~给出了子加群为理想的充分必要条件.
\end{col}

关于商结构, 我们有如下的一些定理.

\begin{thm}[第一同构定理]
    设$\varphi:G\to H$是群的满同态, 那么一定有$G/\ker\varphi$与$H$同构, 且同构映射$\overline\varphi$使得以下图表交换
    \[\begin{tikzcd}
        G \ar[d, "\pi"]\ar[dr, "\varphi"] & \\
        G/\ker\varphi \ar[r, dashed, "\overline\varphi"] & H
    \end{tikzcd}\]
\end{thm}
\begin{proof}
    定义映射
    \begin{align*}
        \overline{\varphi}:G/\ker\varphi&\to H\\
        a\ker\varphi&\mapsto\varphi(a)
    \end{align*}
    对$b\in G$满足$a\ker\varphi=b\ker\varphi$, 有$b^{-1}a\in\ker\varphi$,
    \[b\ker\varphi\mapsto\varphi(b)=\varphi(b)\varphi(b^{-1}a)=\varphi(a)\]
    从而$\overline\varphi$是良定义的, 并且易于发现是同态.
    并且由定义, $\overline\varphi$使得上述图表交换, 从而由$\varphi$满知$\overline\varphi$是满射.
    最后我们说明$\overline\varphi$是单射, 如果$\varphi(a)=\varphi(b)$, 那么$\varphi(a^{-1}b)=e$, 从而$a^{-1}b\in\ker\varphi$, 即$a\ker\varphi=b\ker\varphi$.
\end{proof}

\begin{thm}[对应定理]
    设$f:G\to G'$是满的群同态, 那么对$G$包含$\ker f$的子群$H$, 及$G'$的子群$H'$, 有
    \begin{enumerate}[(1)]
        \item $f(H)$是$G'$的子群, $f^{-1}(H')$是$G$包含$\ker f$的子群;
        \item $G$包含$\ker f$的子群与$G'$的子群通过$f$一一对应;
        \item 如果$H\lhd G$, 那么也有$f(H)\lhd G'$;
        \item 如果进一步地$G$是有限群, 那么$|H|=|\ker f||f(H)|$.
    \end{enumerate}
\end{thm}
\begin{proof}
    (1) 注意到$f(h_1)(f(h_2))^{-1}=f(h_1h_2^{-1})$即可.\\
    (2) 只需验证不同的子群对应的子群不同.
    设$H_1\neq H$是$G$包含$\ker f$的子群, 那么至少存在两个陪集$a\ker f$与$b\ker f$分属于两个子群, 此时$f(a)\neq f(b)$.
    反过来也同理.\\
    (3) 设$h\in H$, 那么对$f(a)\in G'$ ($f$满) 有$f(a)f(h)(f(a))^{-1}=f(aha^{-1})\in f(H)$, 从而$f(H)\lhd G'$.\\
    (4) 由第一同构定理即得.
\end{proof}

\begin{ex}
    陈述并证明环的第一同构定理和对应定理. (注意对应定理是理想间的对应)
\end{ex}

\section{乘积}

我们分别讨论群和环上的乘积结构.

\subsection{乘积群}

设$G,G'$是两个群, 一种简单的构造新群的方式是考虑它们的Descartes积, 即在$G\times G'$上定义乘法
\[(g_1,g_1')\cdot(g_2,g_2')=(g_1g_2,g_1'g_2')\]
容易验证这个乘法使得$G\times G'$成为一个群.

\begin{defn}
    上述构造称为$G$与$G'$的\textbf{直积}.
\end{defn}

而另一种更有趣且更重要的构造是子群间的乘积.

\begin{thm}\label{prod group}
    设$H,K$是$G$的子群, 定义映射$f:H\times K\to G,(h,k)\mapsto hk$.
    记$f$的像集为$HK$.
    \begin{enumerate}[\rm (1)]
        \item $f$是单射当且仅当$H\cap K=\{e\}$;
        \item $f$是同态当且仅当$H$中所有的元素与$K$中所有元素交换;
        \item 如果$H$是$G$的正规子群, 那么$HK$是$G$的子群;
        \item $f$是$H\times K$到$G$的同构, 当且仅当$HK=G$, $H\cap K=\{e\}$且$H,K$为$G$的正规子群.
    \end{enumerate}
\end{thm}
\begin{proof}
    (1) $h_1k_1=h_2k_2\iff h_1h_2^{-1}=k_2k_1^{-1}\in H\cap K$, 那么就有$f$是单射当且仅当$H\cap K=\{e\}$.\\
    (2) 注意到$f(h_1h_2,k_1k_2)=h_1h_2k_1k_2$, $f(h_1,k_1)f(h_2,k_2)=h_1k_1h_2k_2$, 两者相等当且仅当$h_2k_1=k_1h_2$, 由任意性, 此即$H,K$中所有元素交换.\\
    (3) 只需验证$HK$中的元素关于除法封闭.
    取$h_1k_1,h_2k_2$, 有
    \[h_1k_1(h_2k_2)^{-1}=h_1((k_1k_2^{-1})h_2^{-1}(k_2k_1^{-1}))(k_1k_2^{-1})\in HK\]
    (4) 又假设可知$f$满且单, 从而是双射.
    而$H,K\lhd G$, 考虑$hkh^{-1}k^{-1}$, 有
    \[(hkh^{-1})k^{-1}=h(kh^{-1}k^{-1})\in H\cap K\]
    所以$hkh^{-1}k^{-1}=e$, 即$h,k$交换.\footnote{这个技巧叫做取\textbf{交换子}.}
    因此$f$是一个同态, 而且是双射, 从而是同构.
\end{proof}

\subsection{环上的乘积}

我们首先类似群定义环的直积.
\begin{defn}
    两个环的\textbf{直积}是它们的Descartes积及其上自然的运算.
\end{defn}

设$\mathfrak{a},\mathfrak{b}$是环$R$的理想.
定义一个新的理想$\mathfrak{a}+\mathfrak{b}=\{a+b\in R|\ a\in\mathfrak{a},b\in\mathfrak{b}\}$, 容易证明这确实是一个理想.
类似于定理~\ref{prod group}, 我们可以定义理想的直和.

\begin{defn}
    设$\mathfrak{a},\mathfrak{b}$是环$R$的理想.
    如果$\mathfrak{a}\cap\mathfrak{b}=\{0\},\mathfrak{a}+\mathfrak{b}=R$, 那么称$R$是$\mathfrak{a}$与$\mathfrak{b}$的\textbf{直和}, 并记$R=\mathfrak{a}\oplus\mathfrak{b}$.
\end{defn}

\begin{ex}\label{unit of ideal}
    证明当$R=\mathfrak{a}\oplus\mathfrak{b}$时, $\mathfrak{a}$与$\mathfrak{b}$均包含单位元.
\end{ex}

按照习题~\ref{unit of ideal}~中的结论, $\mathfrak{a},\mathfrak{b}$可以看成是环.
那么按照定理~\ref{prod group}, 我们有$\mathfrak{a}\times\mathfrak{b}$与$\mathfrak{a}\oplus\mathfrak{b}$作为加群同构, 并且这个同构可以延拓为环同构.
因此, 我们认为两个环的直积和直和是一样的.

环的直积有一个重要的结论, 即中国剩余定理.
\begin{defn}
    设$\mathfrak{a},\mathfrak{b}$是交换环$A$的理想, 如果$\mathfrak{a}+\mathfrak{b}=A$, 那么称$\mathfrak{a},\mathfrak{b}$\textbf{互素}.
\end{defn}

\begin{thm}[中国剩余定理]
    设交换环$A$的理想$\mathfrak{a},\mathfrak{b}$互素, 那么有同构$A/(\mathfrak{a}\cap\mathfrak{b})\simeq A/\mathfrak{a}\times A/\mathfrak{b}$.
\end{thm}
\begin{proof}
    定义同态
    \begin{align*}
        \varphi:A&\to A/\mathfrak{a}\times A/\mathfrak{b}\\
        a&\mapsto (a+\mathfrak{a},a+\mathfrak{b})
    \end{align*}
    对$(x+\mathfrak{a},y+\mathfrak{b})$, 取$a+b=1$及$z=ay+bx$, 就有
    \begin{gather*}
        z\equiv bx\equiv x \pmod{\mathfrak{a}}\\
        z\equiv ay\equiv y \pmod{\mathfrak{b}}
    \end{gather*}
    从而$\varphi(z)=(x+\mathfrak{a},y+\mathfrak{b})$, 即$\varphi$是满射.
    另一方面, $a\in\ker\varphi$当且仅当$a\in\mathfrak{a}\cap\mathfrak{b}$, 所以第一同构定理给出了$A/(\mathfrak{a}\cap\mathfrak{b})\simeq A/\mathfrak{a}\times A/\mathfrak{b}$.
\end{proof}

需要指出的是, 上面有关环的定义与结论都不局限在两项.
特别地, 中国剩余定理也有一般的有限多个理想的形式, 我们陈述整数的版本, 并直接给出一个常用的计算公式:
\begin{thm}
    设$n_1,\cdots,n_m$是两两互素的整数, 那么同余方程组
    \begin{gather*}
        x\equiv a_1\pmod{n_1}\\
        x\equiv a_2\pmod{n_2}\\
        \cdots\\
        x\equiv a_m\pmod{n_m}
    \end{gather*}
    有模$N=n_1\cdots n_m$意义下的唯一解
    \[x\equiv\sum_{i=1}^ma_i\frac{N}{n_i}l_i\pmod{N}\]
    其中$l_i$满足$l_iN/n_i\equiv 1\pmod{n_i}$.
\end{thm}

证明是直接的, 代入计算即可.

\section{生成关系}

首先我们定义由一个集合生成的子群.

\begin{defn}
    设$G$是一个群, 集合$X\subset G$, 称{\bfseries $X$生成的子群}为
    \[\langle X\rangle:=\{a_1^{\varepsilon_1}a_2^{\varepsilon_2}\cdots a_n^{\varepsilon_m}|\ a_i\in X,\varepsilon_i=\pm 1,i=1,2\cdots,m,m\in\mathbb{N}\}\]
    其中的$a_i$一般有重复.
    如果$G=\langle X\rangle$, 则称$G${\bf 由$X$生成}.
    当$X$是有限集时, 称$G$是{\bf 有限生成群}.
\end{defn}

我们考虑由一个元素生成的群.
\begin{defn}
    设$C=\langle a\rangle$, 那么称$C$是\textbf{循环群}.
\end{defn}

循环群的结构是简单的.
\begin{prop}
    设$C$是循环群, 那么$C$的阶数为无穷大时, $C$同构于$\mathbb{Z}$; $C$的阶数为$n$时, $C$同构于$\mathbb{Z}/n\mathbb{Z}$.
\end{prop}

\begin{defn}
    设$G$是群, $a\in G$, 那么定义$a$的\textbf{阶}为$\langle a\rangle$的阶.
\end{defn}

由Lagrange定理可以得到:
\begin{col}
    有限群中元素的阶整除群的阶.
\end{col}

\begin{thm}[Fermat小定理]
    设$p$是素数, 那么对整数$a$有$a^p\equiv a\pmod{p}$.
\end{thm}

\begin{ex}
    证明费马小定理.
\end{ex}

然后我们来定义一个集合生成的理想.
为了方便, 我们只讨论交换环.
\begin{defn}
    设$X$是交换环$A$的子集, 那么$X$生成的理想定义为
    \[\langle X\rangle=\left\{\left.\sum_{i=1}^mr_ia_i\right|\ r_i\in A,a_i\in X,i=1,2,\cdots,m,m\in\mathbb{N}\right\}\]
    当$X$有限时, 称$\langle X\rangle$是有限生成的.
\end{defn}

关于理想的有限生成有两个等价的条件.
第一个是

\begin{defn}
    称交换环$A$满足\textbf{升链条件}, 如果对于任意上升的理想链$\mathfrak{a}_1\subset\mathfrak{a}_2\subset\cdots$, 都存在正整数$n$使得$\mathfrak{a}_n=\mathfrak{a}_{n+1}=\cdots$.
\end{defn}

关于第二个条件, 我们需要回忆偏序关系.\footnote{如果读者跳过了Nother性这一节, 那么偏序关系可以在环的极大理想处学习.}
\begin{defn}
    非空集合$P$上的一个\textbf{偏序关系}$\prec$满足传递性, 自反性与反对称性, 即
    \begin{enumerate}[(1)]
        \item $a\prec b,b\prec c\implies a\prec c$;
        \item $a\prec a$;
        \item $a\prec b,b\prec a\implies a=b$.
    \end{enumerate}
    一个具有偏序关系的集合称为\textbf{偏序集}.
    偏序集$P$上的一个\textbf{极大元}$m$满足对任意$a\in P$, 如果$m\prec a$, 那么$a=m$.
\end{defn}

\begin{prop}
    设$A$是交换环, 则如下三个命题等价:
    \begin{enumerate}[\rm (1)]
        \item $A$满足升链条件;
        \item $A$中任意理想的集合存在极大元 (以包含关系为偏序);
        \item $A$中任意的理想都是有限生成的.
    \end{enumerate}
\end{prop}
\begin{proof}
    $(1)\implies (2)$: 用反证法, 假设$\mathscr{I}$是$A$中一些理想构成的非空集合, 且其中没有极大元.
    我们归纳地构造一列理想列: 取$\mathfrak{a}_1\in\mathscr{I}$; 假定$\mathfrak{a}_n$已经构造, 那么由于$\mathfrak{a}_n$不是极大元, 存在$\mathfrak{a}_{n+1}$使得$\mathfrak{a}_n\subsetneqq\mathfrak{a}_{n+1}$.
    因此$A$中存在严格上升的理想列
    \[\mathfrak{a}_1\subsetneqq\mathfrak{a}_2\subsetneqq\cdots\subsetneqq\mathfrak{a}_n\subsetneqq\cdots\]
    这与升链条件矛盾.\\
    $(2)\implies (3)$: 取理想集
    \[\mathscr{F}=\{\langle X\rangle|\ X\subset A\ \text{有限}\}\]
    那么由假设, $\mathscr{F}$有极大元, 设为$\langle x_1,\cdots,x_n\rangle$.
    断言$A=\langle x_1,\cdots,x_n\rangle$.
    如若不然, 存在$x\in A$使得$x\notin\langle x_1\cdots,x_n\rangle$, 那么$\langle x_1,\cdots,x_n\rangle\subsetneqq\langle x_1,\cdots,x_n,x\rangle$, 这与$\langle x_1,\cdots,x_n\rangle$的极大性矛盾.
    所以$A=\langle x_1,\cdots,x_n\rangle$是有限生成的.\\
    $(3)\implies (1)$: 设$\mathfrak{a}_1\subset\mathfrak{a}_2\subset\cdots$是上升的理想链, 取
    \[\mathfrak{a}=\bigcup_{n\geq 1}\mathfrak{a}_n\]
    容易验证$\mathfrak{a}$是一个理想.
    设$\mathfrak{a}=\langle x_1,\cdots,x_m\rangle$.
    那么每个$x_i$一定属于某个理想$\mathfrak{a}_{n_i}$, 取$N=\max\{n_i|\ i=1,2,\cdots,m\}$, 就有$\langle x_1,\cdots,x_m\rangle\subset\mathfrak{a}_N$.
    那么对任意$n\geq N$, 都有
    \[\langle x_1,\cdots,x_m\rangle=\mathfrak{a}_N\subset\mathfrak{a}_n\subset\mathfrak{a}=\langle x_1,\cdots,x_m\rangle\]
    从而$\mathfrak{a}_n=\mathfrak{a}_N$, 即$A$满足升链条件.
\end{proof}

\begin{defn}
    如果交换环$A$满足升链条件, 那么称$A$是\textbf{Noerther}的.
\end{defn}

\section{例题与习题}

\begin{eg}
    设$p<q$是两个素数, 我们证明$pq$阶群$G$至多只有一个$q$阶子群.
    假设$Q,S$是两个$G$的$q$阶子群, 由于素数阶群都是循环群, 所以他们的交为$\{e\}$ (请读者证明这两个断言).
    对任意$q_1,q_2\in Q$, 有$q_1^{-1}q_2\in S\implies q_1^{-1}q_2=e$, 则$q_1=q_2$, 从而$Q$中的元素分属于不同的$S$的陪集中.
    因此对$G$做陪集分解, $G$至少有$q$个$S$的陪集, 从而$|G|\geq |Q||S|=q^2>pq$, 矛盾.
    所以$G$至多有一个$q$阶子群.
\end{eg}

\begin{eg}
    我们将在本例中计算$\mathbb{Q}$的自同构群.
    设$f\in\Aut\mathbb{Q}$, 我们先验地给出$f(1)=r(r\neq 0)$.
    对于正整数$n$, 通过归纳法可以得到$f(n)=rn$.
    而对负整数$m$, 有
    \[0=f(0)=f(m)+f(-m)\implies f(m)=-f(-m)=-(-rm)=rm\]
    对有理数$p/q$, 我们有
    \[qf(p/q)=\underbrace{f(p/q)+\cdots+f(p/q)}_{{q\text{个}}}=f(p)=pr\]
    从而$f(p/q)=r(p/q)$.
    因此对所有$x\in\mathbb{Q}$有$f(x)=rx$.
    注意到如果$g(x)=sx$是另一个自同构, 那么有$f\circ g(x)=rsx$.
    从而有$\Aut\mathbb{Q}\simeq\mathbb{Q}^*$, 即有理数乘法群.
\end{eg}

\begin{eg}
    我们将在本例中证明第三同构定理, 以演示如何使用第一同构定理.
    第三同构定理断言, 如果$H,N$是$G$的正规子群且$N\subset H$, 那么有
    \begin{equation}
        \frac{G}{H}\simeq\frac{G/N}{H/N}\label{3rd iso thm}
    \end{equation}
    首先需要证明$H/N\lhd G/N$, 这只需要注意到
    \[(gN)(hN)(g^{-1}N)=N(ghg^{-1})NN=ghg^{-1}N\in H/N\]
    即可.
    而定义同态
    \begin{align*}
        \varphi:G/N&\to G/H\\
        gN&\mapsto gH
    \end{align*}
    由于$N\subset H$, 上述定义是良好的.
    我们考虑$\ker\varphi$, 有$\varphi(gN)=H\iff g\in H$, 那么等价于$gN\in H/N$.
    所以$\ker\varphi=H/N$, 第一同构定理给出~\eqref{3rd iso thm}~式.
\end{eg}

\begin{eg}
    我们将在本例中讨论根式理想.
    设$A$是交换环, $\mathfrak{a}$是$A$的理想.
    定义$\sqrt{\mathfrak{a}}=\{a\in A|\ a^n\in\mathfrak{a},\exists n\in\mathbb{N}\}$.
    我们证明$\sqrt{\mathfrak{a}}$是$A$的理想.
    首先对于$a,b\in\sqrt{\mathfrak{a}}$, 设$a^n\in\mathfrak{a},b^m\in\mathfrak{a}$.
    由于$A$是交换环, $A$上二项式定理成立, 从而有
    \begin{align}
        (a-b)^{m+n}&=\sum_{i=0}^{m+n}(-1)^{m+n-i}\binom{m+n}{i}a^ib^{m+n-i}
    \end{align}
    在以上$m+n$个求和项中, $0\leq i\leq n$时$b^{m+n-i}\in\mathfrak{a}$, $n+1\leq i\leq m+n$时$a^i\in\mathfrak{a}$, 所以求和式$(a-b)^{m+n}$在$\mathfrak{a}$中, 即$\sqrt{\mathfrak{a}}$是$A$的子加群.
    其次对于$a\in\sqrt{\mathfrak{a}},r\in A$, 有$(ra)^n=r^na^n\in\sqrt{\mathfrak{a}}$.
    综上可知$\sqrt{\mathfrak{a}}$是$A$的理想.
\end{eg}

\begin{eg}
    我们将在本例讨论环的直和作为\textit{余积}的性质.
    设$\mathfrak{a},\mathfrak{b}$是环$R$的理想, $R=\mathfrak{a}\oplus\mathfrak{b}$.
    假设对环$A$有同态$f:\mathfrak{a}\to A,g:\mathfrak{b}\to A$, 那么存在唯一的同态$\varphi:R\to A$使得下图交换
    \[\begin{tikzcd}
        \mathfrak{a}\ar[dr,"i_1"]\ar[ddr,"f"'] & & \mathfrak{b}\ar[dl,"i_2"']\ar[ddl,"g"]\\
        & R\ar[d, dashed, "\varphi"] & \\
        & A &
    \end{tikzcd}\]
    图中$i_1,i_2$分别是$\mathfrak{a},\mathfrak{b}$的典范嵌入映射.
    事实上, 对$r=a+b$, 定义$\varphi:r\mapsto f(a)+g(b)$.
    那么显然$\varphi$使得图表交换, 只需说明唯一性.
    假设$\psi$也使图表交换, 那么考虑$\varphi-\psi$, 对$r=a+b$有
    \begin{align*}
        \varphi(r)-\psi(r)&=f(a)+g(b)-\psi(a)-\psi(b)\\
        &=f(a)-\psi(i_1(a))+g(b)-\psi(i_2(b))\\
        &=f(a)-f(a)+g(b)-g(b)\\
        &=0
    \end{align*}
    因此$\varphi=\psi$, 即同态是唯一的.
\end{eg}

\begin{ex}
    证明在偶数阶群中, 方程$x^2=e$有偶数个解.
\end{ex}

\begin{ex}
    证明群$G$是Abel群当且仅当$g\to g^{-1}$是$G$的自同构, 即$G\to G$的同构.
\end{ex}

\begin{ex}\label{center}
    设$G$是群, $Z(G)$是$G$的\textbf{中心}, 即$Z(G):=\{z\in G|\ \forall g\in G:gz=zg\}$.
    \begin{enumerate}[(1)]
        \item $Z(G)$是$G$的正规子群;
        \item $G/Z(G)$同构于$G$的自同构群$\Aut(G)$的子群.
        [提示: 考虑{\itshape 内自同构}, 即每个$g$诱导了一个$\mathrm{int}_g:G\to G,x\mapsto gxg^{-1}$.]
    \end{enumerate}
\end{ex}

\begin{ex}
    设$R_1,R_2$是两个环, $p_1:R_1\times R_2\to R_1,p_2:R_1\times R_2\to R_2$是典范投影映射.
    假设对环$A$存在同态$f:A\to R_1,g:A\to R_2$, 那么存在唯一的同态$\varphi:A\to R_1\times R_2$使得下图交换
    \[\begin{tikzcd}
        & A\ar[ddl, "f"'] \ar[ddr, "g"] \ar[d, dashed, "\varphi"] & \\
        & R_1\times R_2\ar[dl, "p_1"] \ar[dr, "p_2"'] & \\
        R_1 & & R_2
    \end{tikzcd}\]
\end{ex}

\begin{ex}
    设$A$是交换环, $X$是$A$的非空子集, 定义$\Ann(X)=\{a\in A|\ \forall x\in X:ax=0\}$.
    证明$\Ann(X)$是$A$的理想.
\end{ex}

\begin{ex}
    设$A$是交换环, $\mathfrak{a},\mathfrak{b}$是$A$的理想, 定义\textbf{乘积理想}
    \[\mathfrak{a}\mathfrak{b}=\left\{\left.\sum_{i=1}^ma_ib_i\right|\ a_i\in\mathfrak{a},b_i\in\mathfrak{b},i=1,2\cdots,m,m\in\mathbb{N}\right\}\]
    证明$\mathfrak{a}\mathfrak{b}\subset\mathfrak{a}\cap\mathfrak{b}$, 并给出严格包含的例子, 并进一步证明$\sqrt{\mathfrak{a}\mathfrak{b}}=\sqrt{\mathfrak{a}\cap\mathfrak{b}}$.
\end{ex}

\begin{ex}
    设$R$是Noether环, $\mathfrak{a}$是$R$的理想, 证明$R/\mathfrak{a}$也是Noether环.
\end{ex}
\chapter{群的更多性质}
我们在本章讨论群的更多的性质.
我们关心群的作用, 以及有限群的分类.
在此之中我们遇到的最重要的定理将会是三条Sylow定理.

\section{群作用}

我们在接下来几节中关注群作用和群作用的一些应用.
首先给出群作用的定义
\begin{defn}
    群$G$在一个集合$S$上的\textbf{作用}是$G$到$S$的置换群的一个同态$G\to\Aut_{\mathsf{Set}}(S)$\footnote{这个记号表示$S$的排列, 与一个群的自同构群区分.}.
    当同态是单态射时, 称作用是\textbf{忠实}的.
\end{defn}

群作用的等价定义是
\begin{defn}
    群$G$在集合$S$上的一个作用是为每个$g\in G$赋予一个映射$\varphi_g:S\to S$, 满足
    \begin{enumerate}[(1)]
        \item $\varphi_g\circ\varphi_h=\varphi_{gh}$;
        \item $\varphi_e=\mathrm{id}_S$.
    \end{enumerate}
\end{defn}

\begin{sym}
    为简单起见, 在不会引起混淆时一般将$\varphi_g(a)$记作$ga$.
\end{sym}

最简单的群作用是群在自身的作用, 这样的作用有两种.
\begin{defn}
    群$G$中的元素$a$在$G$自身的一个\textbf{左平移}为$g\mapsto ag$;
    $a$在$G$自身的一个\textbf{内自同构}或\textbf{共轭作用}为$g\mapsto aga^{-1}$.
\end{defn}

\begin{ex}
    验证左平移和内自同构都是群作用.
\end{ex}

通过群作用, 我们可以得到如下一条基本的定理
\begin{thm}[Cayley]
    任意有限群都同构于某个置换群的子群.
\end{thm}
\begin{proof}
    考虑$G$的左平移.
    对$a,b\in G$, $ag=bg$可以推出$a=b$, 所以不同的元素给出不同的变换.
    因此$G$是忠实的, 从而$G$同构于$\Aut_{\mathsf{Set}}(G)$的某个子群.
    将$G$的元素一一对应于$\{1,\cdots,n\}$ ($n$为$G$的阶), 那么$\Aut_{\mathsf{Set}}(G)\simeq S_n$, 可以得到$G$同构于$S_n$的子群.
\end{proof}

\begin{defn}
    设群$G$作用在$S$上, 定义$S$上的等价关系$\sim$为$a\sim b$当且仅当存在$g\in G$使得$ga=b$.
    定义该群作用的\textbf{轨道}为$\sim$的等价类.
    如果$S$上仅有一条轨道, 那么称群作用是\textbf{可迁}的.
\end{defn}

\begin{defn}
    设群$G$作用在$S$上, $s\in S$.
    定义$s$的\textbf{稳定化子}为$G_s:=\{g\in G|\ gs=s\}$.
\end{defn}

\begin{lem}
    设群$G$作用在$S$上, $s\in S$.
    如果$t=gs$, 那么$G_t=gG_sg^{-1}$.
\end{lem}
\begin{proof}
    平凡计算.
\end{proof}

\begin{prop}[计数公式]
    设有限群$G$作用在有限集合$S$上, $s\in S$.
    用$O_s$记$s$所在的轨道, 那么有$|O_s||G_s|=|G|$.
\end{prop}
\begin{proof}
    轻微滥用记号, 用$G/G_s$表示$G_s$所有左陪集的集合.
    定义映射
    \begin{align*}
        \varphi:O_s&\to G/G_s\\
        t=gs&\mapsto gG_s
    \end{align*}
    我们验证上述映射是良定义的: 如果另有$t=g's$, 那么$g's=gs\implies g^{-1}g'\in G_s$, 即$gG_s=g'G_s$.
    显然$\varphi$是满射.
    如果$gG_s=g'G_s$, 那么可以得出$g^{-1}g'\in G_s$, 从而$gs=g's$, 即$\varphi$是单射.
    因此$|O_s||G_s|=|G/G_s||G_s|=|G|$.
\end{proof}

我们给出共轭作用的一些应用.
共轭作用的轨道也称为\textbf{共轭类}, 在有限群中, 将所有共轭类的元素个数相加可以得到群的阶数, 这样我们就得到了
\begin{prop}
    有限群$G$的{\bf 类方程}为
    \[|G|=\sum_{O\text{是共轭类}}|O|\]
    特别地, 单位元的共轭类$|O_e|=1$, 从而方程也能写成
    \[|G|=1+\sum_{O\text{是}e\text{以外的共轭类}}|O|\]
\end{prop}

\begin{defn}
    有限群$G$称为{\bf $p$--群}, 如果$G$的阶数是$p$的方幂.
\end{defn}

我们通过类方程给出一些简单的$p$--群的结构.
回忆我们在习题~\ref{center}~中定义了群的中心, 它包含了与群中所有元素交换的元素.
\begin{prop}\label{center of p group}
    $p$--群的中心至少有$p$个元素.
\end{prop}
\begin{proof}
    设$G$是$p$--群.
    注意到中心$Z(G)$中的元素的共轭类仅包含本身, 所以类方程写作
    \begin{equation}
        |G|=\sum_{x\in Z(G)}1+\sum_{O\text{是}G\backslash Z(G)\text{中元素的共轭类}}|O|\label{class equation for p group}
    \end{equation}
    注意到$Z(G)$之外的元素轨道长度一定大于$1$, 而由计数公式, 轨道长度一定整除$|G|=p^n$, 所以长度一定是$p$的倍数.
    因此$\sum_{O\text{是}G\backslash Z(G)\text{中元素的共轭类}}|O|$被$p$整除.
    而~\eqref{class equation for p group}~左端为$p^n$, 所以$\sum_{x\in Z(G)}1$被$p$整除, 且至少是$p$, 即$|Z(G)|\geq p$.
\end{proof}

对于$p^2$阶群, 还有更强的结论
\begin{prop}
    $p^2$阶群是Abel群.
\end{prop}
\begin{proof}
    设群$G$满足$|G|=p^2$.
    由命题~\ref{center of p group}, $|Z(G)|\geq p$, 从而$|Z(G)|=p$或$|Z(G)|=p^2$.
    对于后一种情况, 命题得证.
    对于前一种情况, 我们考虑一个$x\notin Z(G)$, 取$Z_x=\{y\in G|\ xy=yx\}$\footnote{这个群叫做中心化子.}, 容易验证$Z_x$是$G$的一个子群.
    又因为$x\notin Z(G)$, 所以$Z_x$真包含$Z(G)$, 从而$|Z_x|=p^2$.
    而这说明$x$与$G$中所有元素交换, 有$x\in Z(G)$, 矛盾.
    因此$G$是Abel群.
\end{proof}

\begin{ex}
    证明$p^2$阶群是循环群或者是两个$p$阶群的直积.
\end{ex}

\section{单群}

\begin{defn}
    如果群$G$没有非平凡的正规子群, 那么称$G$为\textbf{单群}.
\end{defn}

比较简单的情况是Abel群的情况.
\begin{prop}
    Abel群$G$是单群当且仅当$G$是素数阶循环群.
\end{prop}
\begin{proof}
    注意到Abel群的任意子群都是正规的, 所以$G$是单群等价于$G$没有非平凡子群.
    由Lagrange定理, $G$是素数阶循环群时$G$没有非平凡子群.
    反过来, $\mathbb{Z}$不是单群;
    如果$G$不是循环群, 那么存在$g\in G\backslash\{e\}$使得$\langle g\rangle\subsetneqq G$;
    如果$G$是循环群而不是素数阶的, 设$|G|=mn$, $G=\langle g\rangle$, 那么$\langle g^m\rangle\subsetneqq G$.
    综上可知命题成立.
\end{proof}

而对一般的有限群来说, 另一个常见的结论是
\begin{thm}\label{A_n simple}
    设$n\geq 5$, 那么交错群$A_n$是单群.
\end{thm}

我们首先需要两个引理.
\begin{lem}
    $A_n$由$3$--循环生成.
\end{lem}
\begin{proof}
    $A_n$中的元素都可以写成偶数个$2$--循环的乘积.
    而对两个$2$--循环$(ij),(rs)$有$(ij)(rs)=(ijr)(jrs)$, 所以$3$--循环生成了$A_n$.
\end{proof}

\begin{lem}
    $n\geq 5$时$A_n$中的$3$--循环两两共轭.
\end{lem}
\begin{proof}
    设$(ijk),(i'j'k')$是两个$3$--循环, 那么存在一个置换$\sigma$使得$\sigma(i)=i',\sigma(j)=j',\sigma(k)=k'$.
    如果$\sigma$是偶置换, 那么就有$\sigma(ijk)\sigma^{-1}=(i'j'k')$, 从而$(ijk),(i'j'k')$共轭.
    如果$\sigma$是奇置换, 由于$n\geq 5$, 存在与$i,j,k$不同的$r,s$, 那么用$\sigma\cdot(rs)$代替$\sigma$, 仍然得到$(ijk),(i'j'k')$共轭.
\end{proof}

\begin{proof}[定理~\ref{A_n simple}~的证明]
    由前面的两个引理, 我们只需要证明$A_n$的任意非平凡正规子群$N$均包含一个$3$--循环即可.

    设$\sigma$是$\mathrm{id}$之外不动点最多的置换.
    考虑$\langle\sigma\rangle$作用下的轨道, 那么存在轨道其中含有超过一个元素.
    假设除了一个元素构成的轨道外, 所有轨道都只有两个元素.
    由于$\sigma$是偶置换, 所以至少有两个这样的轨道, 那么$\sigma=(ij)(rs)\cdots$.
    设$k\neq i,j,r,s$, $\tau=(krs)$, 取$\sigma'=\tau\sigma\tau^{-1}\sigma^{-1}$.
    那么简单计算可以得到$\sigma'(i)=i,\sigma'(j)=j$, 并且对$t\neq i,j,k,r,s$, 如果$t$被$\sigma$固定, 那么也被$\sigma'$固定.
    因此$\sigma'$有更多的不动点, 矛盾.

    由上述论证, $\langle\sigma\rangle$的轨道中至少存在一个有至少$3$个元素, 设轨道为$O=\{i,j,k,\cdots\}$.
    如果$\sigma$不是$3$--循环, 那么$O$中至少还有两个元素, 否则$\sigma$中包含$(ijkr)$, 是一个奇置换.
    因此$\sigma$移动$i,j,k$以外的$r,s$, 同理地取$\tau=(krs)$及$\sigma'=\tau\sigma\tau^{-1}\sigma^{-1}$, 那么$\sigma'$固定$i,j$且固定$i,j,k,r,s$以外的其他不动点, 仍然矛盾.
    综合以上两点, 可以知道$\sigma$是一个$3$--循环.
\end{proof}

\begin{ex}
    证明$A_4$不是单群.
\end{ex}

\section{Sylow定理}

有限群理论中一个重要的工具是我们即将陈述的三个Sylow定理.
本节中的群均默认是有限群.
\begin{defn}
    设素数$p$整除群$G$的阶, 那么群$G$的一个\textbf{Sylow $p$--子群}是一个$p^n$阶子群, 其中$n$是$p$整除$|G|$的最高次幂.
\end{defn}

\begin{thm}[Sylow第一定理]
    设素数$p$整除群$G$的阶, 那么群$G$中存在Sylow $p$--子群.
\end{thm}

\begin{thm}[Sylow第二定理]
    设$H$是$G$的$p$--子群, $P$是$G$的Sylow $p$--子群, 那么存在$a\in G$使得$H\subset aPa^{-1}$.
\end{thm}

\begin{col}
    Sylow $p$--子群两两共轭.
\end{col}

\begin{thm}[Sylow第三定理]
    设$|G|=p^nm,(p,m)=1$, 那么$G$的Sylow $p$--子群的个数整除$m$, 且模$p$余$1$.
\end{thm}

Sylow定理的证明比较复杂, 我们将其留在附录中.
读者也可以阅读~\parencite[pp.\ 33--36]{Lang}.
Sylow定理的应用十分重要, 我们给出几个例子.

\begin{eg}
    第零个例子是关于如何分类低阶有限群的.
    比如我们分类$4$阶群, 这用不到Sylow定理:
    按照Lagrange定理, 群中元素的阶只能是$1,2,4$其一.
    如果群中存在$4$阶元, 那么这个群是循环群.
    不然群中除单位元外都是$2$阶元, 设群$G:=\{e,a,b,c\}$.
    那么$\{e,a\}\lhd G,\{e,b\}\lhd G$, 并且有$ab=c$, 所以$G\simeq\{e,a\}\times\{e,b\}$.
    因此$4$阶群同构于$C_4$或$C_2\times C_2$.
\end{eg}

\begin{eg}
    我们证明$15$阶群是循环群.
    设$|G|=15$, 考虑Sylow $3$--子群与Sylow $5$--子群.
    由Sylow第三定理, Sylow $3$--子群的个数整除$5$, 且模$3$余$1$, 因此个数只能是$1$.
    由Sylow第二定理, 这说明Sylow $3$--子群是正规的, 设为$H\lhd G$.
    同理Sylow $5$--子群也是正规的, 设为$K\lhd G$.
    而显然$H\cap K=\{e\}$, 所以$G\simeq H\times K$.
    由于$H$, $K$是阶数互素的素数阶循环群, 所以$H\times K$是循环群 (请读者验证), 即$G$是循环群.
\end{eg}

\begin{eg}
    我们证明$72$阶群不是单群.
    首先有$72=2^3\times 3^2$.
    设群$G$的阶为$72$.
    由Sylow第一定理, $G$存在Sylow $3$--子群, 并且由Sylow第三定理, Sylow $3$--子群的个数整除$8$而模$3$余$1$, 从而为$1$或$4$.
    如果Sylow $3$--子群恰好只有一个, 那么由Sylow第二定理可知它是正规的, 从而$G$有非平凡正规子群;
    如果Sylow $3$--子群有四个, 那么由Sylow第二定理可知$G$的共轭作用是这四个子群上的一个可迁作用, 从而诱导了一个同态$\varphi:G\to S_4$.
    由于$|S_4|=24$, 由对应定理可知$\ker\varphi$的阶至少为$3$, 而$\ker\varphi\lhd G$, 所以$G$有非平凡正规子群.
    综上, $G$一定不是单群.
\end{eg}

\section{例题与习题}

\begin{eg}
    我们证明$2n$阶群有阶为$n$的正规子群.
    由Cayley定理, 不妨设$G$是$S_{2n}$的$2n$阶子群.
    如果$G$中存在一个奇置换, 那么$G$中奇置换与偶置换一定一样多 (请读者验证), 那么$A_{2n}\cap G$就是$G$的$n$阶正规子群.
    因此我们只需要找一个奇置换.
    注意到对任意$g\in G\backslash\{\mathrm{id}\}$, $g$没有不动点, 且若$g$的阶为$d$, 那么任意一个元素在$\langle g\rangle$作用下的轨道长为$d$.
    因此$\langle g\rangle$的轨道是$2n/d$个$d$元集, 从而$g$的符号为$(-1)^{2n-2n/d}=(-1)^{2n/d}$.
    而$G$中阶数超过$3$的元素一定有偶数个 (考虑对应$a\mapsto a^{-1}$), $G$中单位元是$1$阶的, 所以$G$中一定存在$2$阶元, 此时它的符号为$(-1)^n=-1$, 从而找到一个奇置换.
\end{eg}

\begin{eg}
    我们将在本例中证明\textit{Burnside引理}.
    设有限群$G$作用在有限集合$X$上, 记$S^g:=\{s\in X|\ gs=s\}$为$g$的不动点集, $X$在$G$的作用下有$n$条轨道, 那么Burnside引理断言
    \[n=\frac{1}{|G|}\sum_{g\in G}|S^g|\]
    证明用到了交换求和号的技巧.
    我们有
    \begin{align*}
        \sum_{g\in G}|S^g|&=\sum_{g\in G}\sum_{s\in X,gs=s}1=\sum_{s\in X}\sum_{g\in G,gs=s}1\\
        &=\sum_{s\in X}|G_s|=\sum_{s\in X}\frac{|G|}{|O_s|}\\
        &=\sum_{O\text{是轨道}}\sum_{s\in O}\frac{|G|}{|O|}=\sum_{O\text{是轨道}}|G|\\
        &=n|G|
    \end{align*}
\end{eg}

\begin{eg}
    我们证明$n\geq 5$时, $S_n$没有$n!/4$阶子群.
    假设存在子群$G$使得$|G|=n!/4$.
    如果$G$中不存在奇置换, 有$G\subset A_n$, 那么$G$的阶是$A_n$的一半, 从而$G\lhd A_n$, 与$A_n$的单性矛盾.
    如果$G$中存在奇置换, 那么$G'=G\cap A_n$阶为$n!/8$.
    考虑$A_n$在$G'$在$A_n$中的陪集类上的左平移作用: 陪集类中有$4$个元素, 从而这个作用给出一个同态$\varphi:A_n\to S_4$.
    而$n\geq 5$时$|A_n|/|S_4|>1$, 由对应定理知$\ker\varphi$非平凡, 从而$\ker\varphi\lhd A_n$, 与$A_n$的单性矛盾.
    因此$G$是不存在的.
\end{eg}

\begin{ex}
    给定一个正整数$n$, 证明互不同构的$n$阶群只有有限个.
\end{ex}

\begin{ex}
    设有限群$G$可迁地作用在有限集$X$上, $N\lhd G$, 证明$X$在$N$的作用下每个轨道有同样多的元素.
\end{ex}

\begin{ex}
    分类$10$阶群.
\end{ex}
\chapter{交换环}\label{ring}
在第~\ref{structures}~章讨论了环的理想之后, 我们开始具体地讨论交换环的结构.

\section{理想与整环}

\subsection{素理想与极大理想}

\begin{defn}
    称环$A$的子集$S$为\textbf{乘闭子集}, 如果$S$满足$0\notin S,1\in S$, 且对$x,y\in S$有$xy=S$.
\end{defn}

我们首先定义两种重要的理想.
\begin{defn}
    设$\mathfrak{p}$是环$A$的真理想, 如果$\mathfrak{p}$满足对$a,b\in R$, $ab\in\mathfrak{p}$可以推出$a\in\mathfrak{p}$或$b\in\mathfrak{p}$, 那么称$\mathfrak{p}$为\textbf{素理想}.
    所有素理想的集合记为$\Spec{A}$.
\end{defn}

素理想有一种等价的刻画:
\begin{prop}
    设$\mathfrak{p}$是环$A$的理想, 那么以下命题等价:
    \begin{enumerate}[\rm (1)]
        \item $\mathfrak{p}$是素理想;
        \item $R\backslash\mathfrak{p}$是乘闭的;
    \end{enumerate}
\end{prop}
\begin{proof}
    仅仅是重述了一遍素理想的定义.
\end{proof}

\begin{defn}
    设$\mathfrak{m}$是环$A$的真理想, 如果对任意真理想$\mathfrak{a}$, $\mathfrak{m}\subset\mathfrak{a}$可以推出$\mathfrak{m}=\mathfrak{a}$, 那么称$\mathfrak{m}$为$A$的一个\textbf{极大理想}.
    换言之, 极大理想是环$A$的真理想以包含关系为偏序的极大元.
    所有极大理想的集合记为$\MaxSpec{A}$.
\end{defn}

我们在本小节需要证明的中心结论是极大理想的存在性.

\begin{lem}[Zorn]
    设$(X,\prec)$是非空偏序集, 如果$X$中任意一条链均有上界, 那么$X$中存在极大元.
\end{lem}
\begin{proof}
    Zorn引理等价于选择公理, 参阅~\parencite[附录~2.2]{Lang}.
\end{proof}

\begin{prop}
    设$\mathfrak{a}$是$A$的一个真理想, 那么存在极大理想$\mathfrak{m}$使得$\mathfrak{a}\subset\mathfrak{m}$.
\end{prop}
\begin{proof}
    定义$\mathscr{I}$是$A$的所有包含$\mathfrak{a}$的真理想的集合.
    那么$\mathfrak{a}\in\mathscr{I}$, $\mathscr{I}$非空.
    对于$\mathscr{I}$中任意一条链
    \[\mathfrak{a}_1\subset\mathfrak{a}_2\subset\cdots\]
    考虑$\bigcup_{n\geq 1}\mathfrak{a}_n$, 容易验证它构成一个理想.
    我们需要验证$\bigcup_{n\geq 1}\mathfrak{a}_n$是一个真理想, 否则$1\in\bigcup_{n\geq 1}\mathfrak{a}_n$, 那么对某个$\mathfrak{a}_i$有$1\in\mathfrak{a}_i$, 矛盾.
    所以$\bigcup_{n\geq 1}\mathfrak{a}_n$是这条链的上界.
    因此$\mathscr{I}$满足Zorn引理的条件, 其中存在极大元$\mathfrak{m}$.
    断言$\mathfrak{m}$是极大理想:
    如果$\mathfrak{b}$满足$\mathfrak{m}\subset\mathfrak{b}$, 那么$\mathfrak{b}\in\mathscr{I}$, 从而由$\mathfrak{m}$在$\mathscr{I}$中的极大性知$\mathfrak{b}=\mathfrak{m}$.
    因此$\mathfrak{m}$是极大理想, 且包含$\mathfrak{a}$.
\end{proof}

\begin{col}
    环$A$中存在极大理想.
\end{col}

\begin{rem}
    我们强调我们处理的都是含幺交换环, 如果环不含幺元, 那么极大理想很有可能就不存在了, 见下面的例子.
\end{rem}

\begin{eg}
    考虑在$\mathbb{Q}$上赋予平凡乘法, 即对任意$a,b\in\mathbb{Q}$有$ab=0$.
    那么此时$\mathbb{Q}$构成一个不含幺元的交换环.
    假设$\mathbb{Q}$有一个极大理想$\mathfrak{a}$, 那么由对应定理, $\mathbb{Q}/\mathfrak{a}$没有非平凡理想.
    因此$\mathbb{Q}/\mathfrak{a}$没有非平凡子群, 从而是单群, 但单的Abel群只有素数阶循环群, 不妨设$\mathbb{Q}/\mathfrak{a}\simeq\mathbb{Z}/p\mathbb{Z}$.
    取$a\notin\mathfrak{a}$, $a=pb$, 由Lagrange定理可知$p(b+\mathfrak{a})=\mathfrak{a}$, 这与$a\notin\mathfrak{a}$矛盾.
    所以$\mathfrak{a}$不是极大理想.
\end{eg}

\subsection{整环与域}

\begin{defn}
    (不一定交换的) 环$R$的\textbf{零因子}定义为满足存在元素与其相乘为$0$的元素.
\end{defn}

\begin{defn}
    \textbf{整环}是含幺交换无零因子的环.
\end{defn}

通过整环可以构造出一个域.
类似通过$\mathbb{Z}$构造$\mathbb{Q}$的方法, 我们定义整环的\textit{商域}如下.
\begin{defn}
    设$A$是整环, 在$A\times A$上定义等价关系
    \[(r_1,s_1)\sim(r_2,s_2):\iff r_1s_2=r_2s_1\]
    将等价类记为$[r/s]$, 那么$A\times A/\sim$构成一个域, 称为$A$的\textbf{商域}, 并记为$\Quot{A}$.
    $A$可以看作$\Quot{A}$的一个子环, 同构映射由$a\mapsto [a/1]$给出.
\end{defn}

\begin{ex}
    证明一个域的商域是其自身.
\end{ex}

整环与素理想之间可以通过商环建立起联系.
\begin{prop}
    设$\mathfrak{p}$是环$A$的理想, 那么$\mathfrak{p}$是素理想当且仅当$A/\mathfrak{p}$是整环.
\end{prop}
\begin{proof}
    假设$A/\mathfrak{p}$是整环, 那么$ab\in\mathfrak{p}$推出$ab+\mathfrak{p}$, 而$ab+\mathfrak{p}=(a+\mathfrak{p})(b+\mathfrak{p})$, 且对满足$(a+\mathfrak{p})(b+\mathfrak{p})=\mathfrak{p}$的$a,b$一定有$a+\mathfrak{p}=\mathfrak{p}$或$b+\mathfrak{p}=\mathfrak{p}$, 即$a\in\mathfrak{p}$或$b\in\mathfrak{p}$.
    反过来如果$\mathfrak{p}$是素理想, 那么$ab+\mathfrak{p}=\mathfrak{p}$推出$ab\in\mathfrak{p}$, 就有$a\in\mathfrak{p}$或$b\in\mathfrak{p}$, 从而得到$a+\mathfrak{p}=\mathfrak{p}$或$b+\mathfrak{p}=\mathfrak{p}$.
\end{proof}

\begin{prop}\label{maximal ideal and field}
    设$\mathfrak{m}$是环$A$的理想, 那么$\mathfrak{m}$是极大理想当且仅当$A/\mathfrak{m}$是域.
\end{prop}

\begin{lem}
    一个整环是域当且仅当其只有平凡理想.
\end{lem}
\begin{proof}
    设$k$是整环.
    如果$k$是域, 那么$k$的理想$\mathfrak{a}$要么是零理想, 要么存在非零元$a\in\mathfrak{a}$, 那么$1=a^{-1}a\in\mathfrak{a}$, 从而$\mathfrak{a}=k$.
    如果$k$只有平凡理想, 那么对任意$a\neq 0$有$\langle a\rangle=k$, 从而$1\in\langle a\rangle$, 即$a$可逆.
\end{proof}

\begin{proof}[命题~\ref{maximal ideal and field}~的证明]
    假设$\mathfrak{m}$是极大理想, 那么由对应定理, $A/\mathfrak{m}$只有平凡理想, 从而由引理知$A/\mathfrak{m}$是域.
    反过来, 如果$A/\mathfrak{m}$是域, 那么$A/\mathfrak{m}$只有平凡理想, 从而由对应定理, $A$中不存在更大的理想包含$\mathfrak{m}$, 即$\mathfrak{m}$是极大理想.
\end{proof}

\begin{col}\label{maximal implies prime}
    极大理想都是素理想.
\end{col}
\begin{proof}
    域都是整环.
\end{proof}

\section{三种特殊的整环}

\subsection{唯一分解整环}

我们在本小节将推广$\mathbb{Z}$上的唯一分解性, 得到一类具有唯一分解性的整环.
为此, 我们将给出更广泛的整除与唯一分解的定义.

\begin{defn}设$A$是整环.
    \begin{enumerate}[(1)]
        \item 设$u\in A$满足存在$v\in A$使得$uv=1$, 那么称$u$是一个\textbf{单位}.
        \item 设$f,g\in A$满足存在$h\in A$使得$f=gh$, 那么称$g$\textbf{整除}$f$, 并记$g|f$. 此时称$g$是$f$的\textbf{因子}, $f$是$g$的\textbf{倍元}.
        \item 如果$f|g$且$g|f$, 那么称$f$和$g$\textbf{相伴}, 此时存在单位$u$使得$f=ug$.
    \end{enumerate}
\end{defn}

\begin{defn}
    设$A$是整环.
    \begin{enumerate}[(1)]
        \item 如果$f\in A$满足$f=gh$且$g,h$都不是单位, 那么称$f$是\textbf{可约的}, 否则称$f$是\textbf{不可约的}.
        \item 设$f\in A$, 称$f$可以分解为不可约元的乘积, 如果$f=f_1f_2\cdots f_n$, 其中$f_i$都是不可约元.
        \item 设$f\in A$, 称$f$唯一分解为不可约元的乘积, 如果对两个分解
        \[f=f_1f_2\cdots f_l=g_1g_2\cdots g_m\]
        有$l=m$, 且适当调整顺序之后有$f_i$与$g_i$相伴.
    \end{enumerate}
\end{defn}

\begin{defn}
    整环$A$称为是\textbf{唯一分解整环 (UFD)}, 如果$A$中的任意非零且非单位的元素都可以唯一分解为不可约元的乘积.
\end{defn}

在整数中, 素数具有性质$p|ab\implies p|a$或$p|b$, 依此我们可以类似地在整环上定义\textit{素元}的概念.

\begin{defn}
    设$A$是整环, 如果$p\in A$满足对任意$a,b\in A$有$p|ab\implies p|a$或$p|b$, 则称$p$是\textbf{素元}.
\end{defn}

\begin{lem}\label{prime implies irreducible}
    整环上的素元都是不可约元.
\end{lem}
\begin{proof}
    假设$p=ab$且$a,b$都不是单位, 那么$p|ab$, 得出$p|a$或$p|b$, 不妨设前者成立, 那么$a|p$且$p|a$, 可知$p,a$相伴, 从而$b$是单位, 矛盾.
    所以$p$不可约.
\end{proof}

\begin{prop}\label{UFD}
    设$A$是整环, 那么$A$是唯一分解整环的充分必要条件是
    \begin{enumerate}[(1)]
        \item $A$中的每个非零, 非单位的元素都可以分解为不可约元的乘积;
        \item $A$中每个不可约元都是素元.
    \end{enumerate}
\end{prop}
\begin{proof}
    必要性: 设$p|ab$, 进一步设$ab=pr$, 那么作不可约元的分解有
    \[a_1\cdots a_lb_1\cdots b_m=pr_1\cdots r_n\]
    由分解的唯一性, $p$必然与某个$a_i$或者$b_i$相伴, 即$p|a$或$p|b$.
    因此$p$是素元.\\
    充分性: 设$f\in A$有分解
    \[f_1f_2\cdots f_l=g_1g_2\cdots g_m\]
    我们对$\max\{l,m\}$用归纳法.
    $\max\{l,m\}=1$时有$f_1=g_1$, 无需证明.
    假设$\max\{l,m\}=n$, 不妨设$m=n$, 那么有
    \[f_1|g_1g_2\cdots g_n\]
    由于$f_1$是素元, 一定存在某个$g_i$使得$f_1|g_i$, 而$g_i$是不可约元, 所以$f_1$与$g_i$相伴.
    那么设$f_1=ug_i$, 在分解中约去$f_1$与$g_i$后得到
    \[f_2\cdots f_l=ug_1\cdots\widehat{g_i}\cdots g_m\]
    此时两侧不可约元个数最大值为$n-1$, 由归纳假设有$l-1=m-1$, 即$l=m$, 且调整顺序后不可约元对应相伴.
    因此$A$是唯一分解整环.
\end{proof}

在唯一分解整环中, 可以定义两个元素的最大公因子和最小公倍式.
\begin{defn}设$A$是唯一分解整环, $a_1,\cdots,a_n\in A$.
    \begin{enumerate}[(1)]
        \item $a_1,\cdots,a_n$的\textbf{最大公因子}定义为满足$d|a_i(i=1,\cdots,n)$, 且对任意$d'|a_i(i=1,\cdots,n)$的$d'$有$d'|d$的$d\in A$, 记为$(a_1,\cdots,a_n)$.
        \item $a_1,\cdots,a_n$的\textbf{最小公倍式}定义为满足$a_i|l(i=1,\cdots,n)$, 且对任意$a_i|l'(i=1,\cdots,n)$的$l'$有$l|l'$的$d\in A$, 记为$[a_1,\cdots,a_n]$.
    \end{enumerate}
\end{defn}

最大公因子和最小公倍式一般来说不唯一, 会相差一个单位.
例如在$\mathbb{Z}$中, $(4,6)$既可以是$2$也可以是$-2$.

\begin{ex}
    证明唯一分解整环$A$中最大公因子和最小公倍式存在, 并且存在单位$u\in A$使得$a_1\cdots a_n=u(a_1,\cdots,a_n)[a_1,\cdots,a_n]$.
\end{ex}

\subsection{主理想整环}

\begin{defn}
    由一个元素生成的理想称为\textbf{主理想}.
    如果整环$A$的每个理想都是主理想, 那么称$A$为\textbf{主理想整环 (PID)}.
\end{defn}

我们希望证明主理想整环是唯一分解整环.
为此, 我们需要建立主理想整环的一些性质:

\begin{prop}\label{PID 1}
    在主理想整环$A$中, $p\in A\backslash\{0\}$, 那么下列命题等价:
    \begin{enumerate}[\rm (1)]
        \item $p$是不可约元;
        \item $\langle p\rangle$是极大理想;
        \item $\langle p\rangle$是素理想;
        \item $p$是素元.
    \end{enumerate}
\end{prop}
\begin{proof}
    $(1)\implies(2)$: 假设$\langle p\rangle\subset\mathfrak{a}\neq A$, 那么由于$A$是主理想整环, $\mathfrak{a}=\langle a\rangle$, 从而$a|p$.
    而$p$是不可约元, 这说明$a$是单位或$a=p$, 即$\mathfrak{a}=\langle p\rangle$, 从而$\langle p\rangle$是极大理想.\\
    $(2)\implies(3)$: 这是推论~\ref{maximal implies prime}.\\
    $(3)\implies(4)$: 设$p|ab$, 那么$ab\in\langle p\rangle$, 从而有$a\in\langle p\rangle$或$b\in\langle p\rangle$, 即$p|a$或$p|b$.\\
    $(4)\implies(1)$: 这是引理~\ref{prime implies irreducible}.
\end{proof}

\begin{prop}\label{PID 2}
    主理想整环$A$的主理想满足升链条件, 即对主理想的升链
    \begin{equation}
        \langle a_1\rangle\subset\langle a_2\rangle\subset\cdots\label{chain of principle ideal}
    \end{equation}
    存在$n\in\mathbb{N}$使得$\langle a_n\rangle=\langle a_{n+1}\rangle=\cdots$.
\end{prop}
\begin{proof}
    在~\eqref{chain of principle ideal}~中取$\mathfrak{a}=\bigcup_{i\geq 1}\langle a_i\rangle$, 那么我们熟悉这一定是一个理想.
    由于$A$是主理想整环, 那么存在$a\in\mathfrak{a}$使得$\mathfrak{a}=\langle a\rangle$.
    由于$a\in\bigcup_{i\geq 1}\langle a_i\rangle$, 设$a\in\langle a_n\rangle$, 那么
    \[\langle a\rangle\subset\langle a_n\rangle\subset\langle a_{n+1}\rangle\subset\cdots\subset\langle a\rangle\]
    从而就有$\langle a_n\rangle=\langle a_{n+1}\rangle=\cdots$.
\end{proof}

\begin{prop}
    主理想整环是唯一分解整环.
\end{prop}
\begin{proof}
    设$A$是主理想整环, 我们证明$A$中非零非单位的元素都能分解为不可约元的乘积.
    否则设存在一个$a$不可以分解为不可约元的乘积, 设$a=a_1b_1$, 其中$a_1$不是不可约元, 不妨设其也不能分解为不可约元的乘积.
    又设$a_1=a_2b_2$, $a_2$不可以分解为不可约元的乘积.
    如此归纳定义得到序列$a_1,a_2,\cdots$, 每一项中后者都整除前者且不与前者相伴, 因此我们得到严格递增的主理想链
    \[\langle a_1\rangle\subsetneqq\langle a_2\rangle\subsetneqq\cdots\]
    这与命题~\ref{PID 2}~矛盾.
    所以$A$中的元素都可以分解为不可约元的乘积.
    而由命题~\ref{PID 1}, $A$中的不可约元都是素元, 那么由命题~\ref{UFD}, 可知$A$是唯一分解整环.
\end{proof}

反过来一般是不成立的.
例如可以证明$\mathbb{Z}[x]$是唯一分解整环 (\parencite[p.\ 182定理2.3]{Lang}), 但是容易发现$\langle 2,x\rangle$不是主理想.

\subsection{Euclid整环}

在本小节我们推广$\mathbb{Z}$上的带余除法.

\begin{defn}
    设$A$是整环, 映射$\delta:A\backslash\{0\}\to\mathbb{N}$, 满足对任意$a,b\in A$, 存在$q,r\in A$使得
    \[a=bq+r\]
    且$r=0$或$\delta(r)<\delta(b)$, 则称$A$为\textbf{Euclid整环}, $\delta$为\textbf{Euclid映射}.
\end{defn}

我们熟知两种Euclid整环$\mathbb{Z}$与$k[x]$.
当$A=\mathbb{Z}$时, Euclid映射就是恒等映射; 当$A=k[x]$时, Euclid映射是多项式的度数.

我们证明本小节最主要的结论:
\begin{prop}
    Euclid整环是主理想整环.
\end{prop}
\begin{proof}
    设$A$是Euclid整环, $\mathfrak{a}$是$A$的理想.
    取集合$S=\{\delta(x)|\ x\in\mathfrak{a}\}$, 那么$S\subset\mathbb{N}$, 由最小数原理, 存在$a\in\mathfrak{a}$使得$\delta(a)=\min{S}$.
    断言$\mathfrak{a}=\langle a\rangle$.
    设$b\in\mathfrak{a}$, 那么存在$q,r\in A$使得$b=aq+r$.
    如果$r\neq 0$, 那么$r=b-aq\in\mathfrak{a}$, 且$\delta(r)<\delta(a)$, 与$\delta(a)=\min{S}$矛盾.
    所以$r=0$, 即$b=aq$, $b\in\langle a\rangle$.
    从而有$\mathfrak{a}=\langle a\rangle$, 即$A$的任意理想是主理想.
\end{proof}

本小节与前一小节证明了如下的包含关系:
\begin{center}
    Euclid整环$\subsetneqq$主理想整环$\subsetneqq$唯一分解整环
\end{center}
而证明这两个包含关系是严格的则超出了本讲义的范围.

\section{例题与习题}

\begin{eg}
    我们证明交换环$A$的诣零根满足
    \begin{equation}
        \sqrt{\{0\}}=\bigcap_{\mathfrak{p}\in\Spec{A}}\mathfrak{p}\label{nilradical}
    \end{equation}
    一方面, 容易验证素理想都是根式理想, 所以$\sqrt{\{0\}}\subset\sqrt{\mathfrak{p}}=\mathfrak{p},\forall\mathfrak{p}\in\Spec{A}$, 即
    \[\sqrt{\{0\}}\subset\bigcap_{\mathfrak{p}\in\Spec{A}}\mathfrak{p}\]
    另一方面, 如果$a\in A\backslash\sqrt{\{0\}}$, 那么$S:=\{1,a,a^2,\cdots\}$是一个乘闭子集, 从而$A\backslash S\in\Spec{A}$, $a\notin\bigcap_{\mathfrak{p}\in\Spec{A}}\mathfrak{p}$, 因此
    \[A\left\backslash\sqrt{\{0\}}\right.\subset A\left\backslash\bigcap_{\mathfrak{p}\in\Spec{A}}\mathfrak{p}\right.\]
    因此有~\eqref{nilradical}~成立.
\end{eg}

\begin{eg}
    我们在本例中证明\textit{素理想躲避引理}.
    设$A$是交换环, $\mathfrak{p}_1,\cdots,\mathfrak{p}_n$是素理想, 理想$\mathfrak{a}$满足
    \[\mathfrak{a}\subset\bigcup_{i=1}^n\mathfrak{p}_i\]
    那么存在$i\in\{1,\cdots,n\}$使得$\mathfrak{a}\subset\mathfrak{p}_i$.

    事实上, 对$n$用归纳法.
    $n=1$时命题显然成立.
    对$n>1$, 考虑集合$A_i:=\mathfrak{a}\left\backslash\bigcup_{j\neq i}\mathfrak{p}_j\right.$.
    如果某个$A_i=\varnothing$, 那么$\mathfrak{a}\subset\bigcup_{j\neq i}\mathfrak{p}_j$, 由归纳假设知命题成立.
    现假设每个$A_i$均非空, 反设命题不成立, 那么取$x_i\in A_i$, 易知$x_i\in\mathfrak{p}_i$.
    考虑$x_1\cdots x_{n-1}+x_n\in\mathfrak{a}$, 当$1\leq i\leq n-1$时,
    \[x_1\cdots x_{n-1}+x_n\in\mathfrak{p}_i\implies x_n\in\mathfrak{p}_i\]
    矛盾; 当$i=n$时,
    \[x_1\cdots x_{n-1}+x_n\in\mathfrak{p}_n\implies\exists x_j\in\mathfrak{p}_n\]
    仍然矛盾.
    因此由归纳法可知命题成立.
\end{eg}

\begin{eg}
    设$A$是整环, 我们证明$A[x]$是主理想整环当且仅当$A$是域.
    熟知$A$是域时$A[x]$是主理想整环.
    反过来, 假设$A[x]$是主理想整环, 对$a\in A\backslash\{0\}$, 考虑理想$\langle a,x\rangle$.
    由于$A[x]$是主理想整环, 设$\langle a,x\rangle=\langle b\rangle$.
    那么$b|a$, 考虑度数可知$b\in A$.
    而$b|x$, 设$b(cx+d)=x$, 比较系数可知$bc=1$, 即$b$可逆.
    所以$\langle a,x\rangle=A[x]$, 那么存在$f(x),g(x)\in A[x]$使得
    \[af(x)+xg(x)=1\]
    令$x=0$有$af(0)=1$, 即$a$可逆, 从而$A$是一个域.
\end{eg}

\begin{eg}
    我们证明$\mathbb{Z}[i]/\langle 1+i\rangle\simeq\mathbb{F}_2$.
    而这只需要观察如下图表:
    \[\begin{tikzcd}
        \mathbb{Z}[x]\ar[r, "x^2+1"]\ar[d, "x+1"] & \mathbb{Z}[i]\ar[d, "i+1"]\\
        \mathbb{Z}\ar[r, "2"] & \mathbb{F}_2
    \end{tikzcd}\]
\end{eg}

\begin{ex}
    设$A$是交换环, 如果$e\in A$满足$e^2=e$, 则称$e$是幂等元.
    \begin{enumerate}[(1)]
        \item 如果$e$是幂等元, 证明$1-e$也是幂等元.
        \item 证明$A\simeq\langle e\rangle\oplus\langle 1-e\rangle$.
    \end{enumerate}
\end{ex}

\begin{ex}
    一个交换环称为\textbf{局部环}, 如果它有唯一的极大理想.
    证明一个交换环是局部环当且仅当它的所有不可逆元构成一个理想.
\end{ex}

\begin{ex}
    交换环$A$上的\textbf{形式幂级数环}$A[[x]]$是所有形如
    \[\sum_{n=0}^\infty a_nx^n\]
    的元素构成的环, 其中加法与乘法与多项式的定义类似.
    \begin{enumerate}[(1)]
        \item 证明$a_0+a_1x+a_2x^2+\cdots$可逆当且仅当$a_0$是单位.
        \item 设$k$是域, 证明$k[[x]]$是局部环.
    \end{enumerate}
\end{ex}

\begin{ex}
    \begin{enumerate}[(1)]
        \item 证明$\mathbb{Z}[\sqrt{-5}]$不是唯一分解整环.
        \item 证明$\mathbb{Z}[\sqrt{-1}],\mathbb{Z}[\sqrt{-2}]$是唯一分解整环.
        \par [提示: 考虑$\mathbb{Z}[\sqrt{-1}]$的\textbf{模}$|a+b\sqrt{-1}|^2=a^2+b^2$, $\mathbb{Z}[\sqrt{-2}]$的模类似定义.]
    \end{enumerate}
\end{ex}
\chapter{域}\label{field}
我们在本章及下一章讨论域这种更强的代数结构.

\section{域扩张}

\begin{defn}
    设$K,L$是域, 满足$K\subset L$, 那么称$L$是$K$的一个\textbf{扩域}, 并记为$L/K$.
\end{defn}

我们首先引进两个记号
\begin{sym}
    设有域扩张$L/K$, $S\subset L$, 那么记$K(S)$为包含$S$的最小的域 (即所有包含$S$的域的交).
    如果$S=\{a_1,\cdots,a_n\}$, 也记$K(S)=K(a_1,\cdots,a_n)$.
\end{sym}

对$L/K$, $L$自然构成了一个$K$--向量空间, 所以我们可以定义
\begin{defn}
    定义域扩张$L/K$的\textbf{度数}为$[L:K]:=\dim_K{L}\in\mathbb{N}\cup\{\infty\}$.
\end{defn}

我们定义几类扩张如下
\begin{defn}
    给定域扩张$L/K$.
    \begin{enumerate}[(1)]
        \item 如果$[L:K]<\infty$, 那么称$L/K$是\textbf{有限扩张}.
        \item 设$a\in L$, 如果$\alpha$是$K[x]$中某个多项式的根, 那么称$\alpha$是$K$上的\textbf{代数元}, 否则称为\textbf{超越元}.
        如果$\alpha$是代数元, 所有满足$f(\alpha)=0$的多项式中次数最低的称为$\alpha$的\textbf{极小多项式}.
        \item 如果$L$中任意一个元素都是代数元, 那么称$L/K$是\textbf{代数扩张}, 否则称为\textbf{超越扩张}.
    \end{enumerate}
\end{defn}

考虑由单个代数元$\alpha$生成的扩域, 我们有如下的引理
\begin{lem}
    设$L/K$, $\alpha\in L$是$K$上的代数元, 有极小多项式$m(x)\in K[x]$, 那么$K(\alpha)\simeq K[x]/\langle m(x)\rangle$, 其中$\langle m(x)\rangle$是$p(x)$生成的理想.
\end{lem}
\begin{proof}
    定义同态
    \begin{align*}
        \varphi:K[x]&\to K(\alpha)\\
        p(x)&\mapsto p(\alpha)
    \end{align*}
    考虑核$\ker\varphi$, 显然$\ker\varphi\neq K[x]$, 且极小多项式$m(x)\in\ker\varphi$.
    由于$K[x]$是主理想整环, $\ker\varphi$单生成, 且生成元整除$m(x)$.
    但容易证明$m(x)$是不可约多项式, 结合$\ker\varphi\neq K[x]$可知生成元与$m(x)$相伴, 从而$\ker\varphi=\langle m(x)\rangle$.
    由第一同构定理即知
    \[K(\alpha)\simeq \frac{K[x]}{\langle m(x)\rangle}\qedhere\]
\end{proof}

关于有限扩张与代数扩张, 有如下的结论
\begin{prop}
    设有域扩张$M/K$, $\alpha\in M$是$K$上的代数元当且仅当$\alpha$包含在$K$的一个有限扩张中.
\end{prop}
\begin{proof}
    一方面, 假设$\alpha$是代数元.
    设$\deg\alpha=n$, 那么$1,\alpha,\cdots,\alpha^{n-1}$是$K(\alpha)$的一组基, $K(\alpha)/K$是有限扩张.
    另一方面, 假设$\alpha$包含在$K$的有限扩张中, 不妨设$[M:K]=n<\infty$.
    那么$1,\alpha,\cdots,\alpha^{n-1},\alpha^n$一定线性相关, 从而$\alpha$是一个多项式的根, 是一个代数元.
\end{proof}

\begin{col}
    任意有限扩张都是代数扩张.
\end{col}

\begin{thm}[望远镜公式]\label{telescope}
    设$K\subset L\subset M$均为有限扩张, 那么有$[M:K]=[M:L][L:K]$
\end{thm}
\begin{proof}
    设$x_1,\cdots,x_m$是$L/K$的一组基, $y_1,\cdots,y_n$是$M/L$的一组基.
    我们考虑$\{x_iy_j\}_{(i,j)\subset [m]\times[n]}$\footnote{$[m]=\{1,\cdots,m\}$, 组合数学中的常用记号.}.
    首先对$a_{ij}\in K$及指标$(i,j)\in R\times S\subset [m]\times[n]$有
    \begin{align*}
        &\sum_{(i,j)\in R\times S}a_{ij}(x_iy_j)=0\\
        \implies&\sum_{j\in S}a_{ij}y_j=0,\ \forall i\in R\\
        \implies&a_{ij}=0,\ \forall (i,j)\in R\times S
    \end{align*}
    所以$x_iy_j$线性无关.
    其次, 显然$M$中的每个元素可以表示为$x_iy_j$的$K$--线性组合, 所以$\{x_iy_j\}_{(i,j)\subset [m]\times[n]}$是$M/L$的一组基.
    从而命题得证.
\end{proof}

通过望远镜公式, 我们可以证明
\begin{thm}
    设$\alpha,\beta$是$K$上的代数元, 那么$\alpha\pm\beta,\alpha\beta,\alpha/\beta(\beta\neq 0)$均为$K$上的代数元.
\end{thm}
\begin{proof}
    考虑扩张链$K\subset K(\alpha)\subset K(\alpha,\beta)$, 两个扩张均为代数扩张, 所以都是有限扩张.
    由定理~\ref{telescope}, $K(\alpha,\beta)/K$是代数扩张.
    而$\alpha\pm\beta,\alpha\beta,\alpha/\beta$均包含在$K(\alpha,\beta)$中, 所以都是代数元.
\end{proof}

\begin{thm}\label{algcoef}
    设$\alpha$是一个由$K$上代数元系数构成的多项式的根, 那么$\alpha$是代数的.
\end{thm}
\begin{proof}
    设
    \[\alpha^n+a_{n-1}\alpha^{n-1}+\cdots+a_0=0\]
    且$a_{n-1},\cdots,a_0$均为$K$上代数元.
    考虑域扩张链
    \begin{align*}
        K&\subset K(a_0)\\
        &\subset K(a_0,a_1)\\
        &\cdots\\
        &\subset K(a_0,\cdots,a_{n-1})\\
        &\subset K(a_0,\cdots,a_{n-1},\alpha)
    \end{align*}
    前$n$步扩张每一步都是添加一个代数元$a_i$, 所以都是有限的, 因此$K$上的扩域$K(a_0,\cdots,a_{n-1})$是有限的.
    而由假设, $\alpha$在$K(a_0,\cdots,a_{n-1})$上代数, 所以最后一步扩张也是有限的.
    因此扩张$K(a_0,\cdots,a_{n-1},\alpha)/K$是有限的, 从而$\alpha$是$K$上代数元.
\end{proof}

\begin{col}\label{alg of alg}
    假设$E/L,L/K$均为代数扩张, 那么$E/K$也是代数扩张.
\end{col}

\section{代数闭包与分裂域}

对于一个代数方程, 我们总希望能够找到一个域使得它``有根''.
而严格地描述这一点则需要定义\textit{分裂域}的概念.

\begin{defn}
    设$K$是域, $S\subset K[x]$, 如果扩域$L/K$使得$S$中的任意$p(x)$在$L$上可以分解为一次因式的乘积 (简称为$p(x)$在$L$中\textbf{分裂})
    \[p(x)=(x-\alpha_1)\cdots(x-\alpha_n)\]
    且$L$由$S$中多项式的根生成, 那么称$L$是$S$在$K$上的\textbf{分裂域}.
\end{defn}

为了更加方便地处理事情, 我们直接``添加域中所有多项式的根''.
这样就定义了\textit{代数闭包}.

\begin{defn}
    设$K$是域, $K$的\textbf{代数闭包}是$k[x]$的分裂域.
\end{defn}

我们期待的结果自然是
\begin{prop}\label{alg closure}
    任意域$K$的代数闭包存在.
\end{prop}

以及
\begin{thm}[同构延拓定理]\label{iso ext thm}
    设$K$是一个域, $S\subset K[x]$是一族多项式, $K'$与$K$同构且$S$在同构映射下的像为$S'$.
    设$E,E'$分别是$S,S'$的分裂域, 那么存在同构$S\to S'$使得下图交换
    \[\begin{tikzcd}
        E\ar[r, "\sim"] & E'\\
        K\ar[u] \ar[r, "\sim"] & K\ar[u]
    \end{tikzcd}\]
\end{thm}

代数闭包的存在性与同构延拓定理的证明较为复杂, 我们将其留在附录 \ref{proof of alg closure} 与 \ref{proof of iso ext thm} 中.

\begin{col}
    设$K$是域, $S\subset K[x]$, 那么$S$的分裂域存在.
\end{col}
\begin{proof}
    注意到$S$中的多项式均在$\overline{K}$中分裂, 那么取$R$为$S$中所有多项式根的集合, $K(R)\subset\overline{K}$即为$S$的分裂域.
\end{proof}

\begin{col}
    任意集合的分裂域在同构意义下唯一.
\end{col}

关于代数闭包, 有一个密切相关的概念是代数闭域:
\begin{defn}
    域$L$被称为是\textbf{代数闭域}, 如果$L[x]$中的任意多项式都在$L$中有根.
\end{defn}
\begin{prop}
    域$K$的代数闭包$\overline{K}$是代数闭域.
\end{prop}
\begin{proof}
    设$p(x)=x^n+a_{n-1}x^{n-1}+\cdots+a_0,\ a_i\in\overline{K}$.
    由于$\overline{K}$由$K[x]$中多项式的根生成, 因此$a_i$均为$K$上的代数元.
    对$p(x)$在某个根$\alpha$, 考虑扩张链
    \[K\subset K(a_0,\cdots,a_{n-1})\subset K(a_0,\cdots,a_{n-1},\alpha)\]
    容易发现两个扩张都是有限的, 所以$\alpha$也是$K$上的代数元, 从而在$\overline{K}$内.
    因此$\overline{K}$是代数闭域.
\end{proof}
这说明在$\overline{K}$中不仅$K[x]$中的多项式分裂, $\overline{K}[x]$中的多项式也分裂, 这是强于代数闭包的定义的.

\section{有限域}

我们在本节讨论元素个数有限的域, 也即\textbf{有限域}.

假设$F$是有限域, 那么$F$一定有正的特征$p>0$.
那么此时素域$\mathbb{F}_p\subset F$, $F$是$\mathbb{F}_p$上的向量空间.
如果$\dim_{\mathbb{F}_p}F=n$, 那么每个坐标分量有$p$种取法, 则$|F|=p^n$.
因此我们得到
\begin{prop}
    有限域$F$的阶为$p^n$, 其中$p=\ch{F}$是素数, $n=[F:\mathbb{F}_p]$.
\end{prop}

相同的论证我们可以得到
\begin{prop}\label{finite subfield}
    设$K,L$分别是$p^n,q^m$元域, 那么$K\subset L$当且仅当$p=q$且$n|m$.
\end{prop}

与分裂域一样, 我们也要讨论有限域的存在性与同构唯一性.

首先我们证明有限域的存在性.
\begin{thm}
    对素数$p$及$q=p^n$, 存在$q$阶有限域.
\end{thm}
\begin{proof}
    取$x^q-x$在$\mathbb{F}_p$上的一个分裂域$L$, 我们证明$L$恰好由$x^q-x$的所有根构成.
    我们先证明$x^q-x$的根构成一个域.
    对根$x,y$, 由$\ch L=p$可知$\binom{q}{k}=0,\ k=1,\cdots,q-1$, 从而
    \begin{align*}
        (x-y)^q&=x^q-y^q\quad(p=2\text{时}1=-1,\ \text{所以均写为减号})\\
        &=x-y
    \end{align*}
    所以$x-y$是$x^q-x$的一个根;
    而当$y\neq 0$时
    \begin{align*}
        \left(\frac{x}{y}\right)^q-\frac{x}{y}&=\frac{x^qy-xy^q}{y^{q+1}}\\
        &=\frac{xy-yx}{y^{q+1}}\\
        &=0
    \end{align*}
    所以$x/y$也是一个根.
    因此$x^q-x$的根在减法与除法下封闭, 构成一个域.
    由于分裂域由根生成, 所以$L$恰好由$x^q-x$的根构成.
    另一方面, 由于$(x^q-x)'=qx^{q-1}-1=-1$, 与$x^q-x$互素, 所以$x^q-x$没有重根.
    因此$|L|=\deg(x^q-x)=q$.
\end{proof}

然后我们证明有限域的唯一性.
\begin{thm}
    两个有限域同构当且仅当他们阶数相同.
\end{thm}
\begin{proof}
    设有限域$F$的阶数为$q$, 我们证明$F$一定是$x^q-x$的分裂域.
    这只需要证明对任意$a\in F$有$a^q=a$即可.
    $a=0$时这是平凡的.
    对$a\in F^*$, 由Lagrange定理, $a^{|F^*|}=1$, 即$a^{q-1}=1$, 从而$a^q=a$.
    因此$F$是$x^q-x$的分裂域, 在同构意义下是唯一的.
\end{proof}

最后我们证明有限域最重要的结论之一
\begin{thm}\label{single generate}
    有限域的乘法群是循环群.
\end{thm}

我们首先需要一个引理
\begin{lem}[多项式的Lagrange定理]
    设$f(x)\in k[x]$, $\deg f(x)=d$, 那么$f(x)=0$在$k$中至多有$d$个根.
\end{lem}
\begin{proof}
    对$d$用归纳法.
    当$d=1$时命题是显然的.
    假设命题对$d=n-1$成立, 那么此时$n$次多项式$f(x)$在$k$上要么没有根, 要么有一个根$\alpha$, 此时存在一个$n-1$次多项式$g(x)$使得$f(x)=(x-\alpha)g(x)$.
    而由归纳假设, $g(x)$至多有$n-1$个根, 所以$f(x)=(x-\alpha)g(x)$至多有$n$个根.
    由归纳原理知命题得证.
\end{proof}

\begin{proof}[定理~\ref{single generate}~的证明]
    设$k$是一个$q$元域, 那么它的乘法群$k^*$阶为$q-1$.
    设$m$是$k^*$中元素的最大值, 并设$\alpha$的阶为$m$.
    那么对任意$\beta\in k^*$, 设其阶为$d$, 则$\alpha\beta$的阶为$[m,d]\leq m$, 从而$d|m$.
    因此$\beta^m=1$, 方程$x^m-1$有至少$q-1$个根, 由多项式的Lagrange定理可知$m\geq q-1$.
    而由群的Lagrange定理知$m|q-1$, 所以$m=q-1$, 即$\langle\alpha\rangle=k^*$.
\end{proof}

\section{例题与习题}

\begin{eg}
    我们证明一个代数闭域一定有无穷多个元素.
    否则假设$K=\{a_1,\cdots,a_n\}$是代数闭域, 我们考察多项式
    \[(x-a_1)(x-a_2)\cdots(x-a_n)+1\]
    $K$中任意元素都不是它的根, 所以它不在$K$上分裂, 与$K$代数闭矛盾.
    因此代数闭域一定有无穷多个元素.
\end{eg}

\begin{eg}
    设$x$是$\mathbb{Q}$上的超越元, $u=x^3/(x+1)$, 我们求$[\mathbb{Q}(x):\mathbb{Q}(u)]$.
    注意到$x$满足$x^3-ux-u=0$, 所以$x$是$\mathbb{Q}(u)$上的代数元.
    断言$u$必然是超越元, 否则如果多项式$f(u)\in\mathbb{Q}[u]$是$u$的零化多项式, 设$\deg f=d$, 那么$(x+1)^df(x^3/(x+1))\in\mathbb{Q}[x]$便是$x$的零花多项式, 矛盾.
    又断言$t^3-ut-u\in\mathbb{Q}(u)[t]$是$x$的极小多项式, 只需要证明$t^3-ut-u$不可约.
    如果$t^3-ut-u$可约的话, 那么一定存在一个$f(u)/g(u)$作为它的根, 但此时有
    \[(f(u))^3-uf(u)(g(u))^2-(g(u))^3=0\]
    与$u$是超越元矛盾.
    因此$t^3-ut-u\in\mathbb{Q}(u)[t]$是$x$的极小多项式, 可以得到$[\mathbb{Q}(x):\mathbb{Q}(u)]=3$.
\end{eg}

\begin{eg}\label{sqrt(2)+sqrt(3)}
    我们证明$\mathbb{Q}(\sqrt{2},\sqrt{3})=\mathbb{Q}(\sqrt{2}+\sqrt{3})$.
    显然$\mathbb{Q}(\sqrt{2}+\sqrt{3})\subset\mathbb{Q}(\sqrt{2},\sqrt{3})$.
    而设$u=\sqrt{2}+\sqrt{3}$, 有$u^2=5+2\sqrt{6},(u^2-5)^2=24$.
    容易验证$(x^2-5)^2-24=x^4-10x^2+1$是不可约的四次多项式, 因此$u$的次数为$4$, 即$[\mathbb{Q}(\sqrt{2}+\sqrt{3}):\mathbb{Q}]=4$.
    而显然$[\mathbb{Q}[\sqrt{2},\sqrt{3}]:\mathbb{Q}]=4$, 所以$[\mathbb{Q}(\sqrt{2},\sqrt{3}),\mathbb{Q}(\sqrt{2}+\sqrt{3})]=1$, 即$\mathbb{Q}(\sqrt{2},\sqrt{3})=\mathbb{Q}(\sqrt{2}+\sqrt{3})$.
\end{eg}


\chapter{Galois理论}
我们在本章讨论有限Galois理论.
Galois理论的中心是Galois对应, 即一个Galois扩张的中间域与其Galois群的子群反序一一对应.
同时作为应用, 我们会证明关于$5$次一般方程不可根式解的Abel--Ruffini定理.

\section{三类域扩张}
我们在本节讨论Galois对应所要求的域扩张, 即Galois扩张及比Galois扩张更一般的正规扩张与可分扩张.

\subsection{正规扩张与可分扩张}

首先我们关心一个多项式在扩域中是否有足够多的根, 与此相关的是正规扩张的概念.

\begin{prop}\label{normal thmdef}
    设$K\subset L$是代数扩张, 那么以下三个命题等价:
    \begin{enumerate}[\rm (1)]
        \item $K[x]$中任何在$L$上有根的多项式$p(x)$在$L[x]$中分裂;
        \item $L$是$K$上某一族多项式的分裂域;
        \item $\overline{K}$的所有固定$K$不动的自同构都将$L$映为$L$.
    \end{enumerate}
\end{prop}

\begin{defn}
    满足命题~\ref{normal thmdef}~中三个等价条件中任意一个的代数扩张称为\textbf{正规扩张}.
\end{defn}

\begin{proof}[命题~\ref{normal thmdef}~的证明]
    我们按照$(1)\implies(2)\implies(3)\implies(1)$的顺序证明它们等价.\\
    $(1)\implies(2)$: 对任意$\alpha\in L$, 由于$L/K$是代数扩张, 因此可以取$\alpha$在$K$上的极小多项式$p_\alpha(x)$, 那么$L$是
    \(S=\{p_\alpha\in K[x]:\ \alpha\in L\}\)
    的分裂域.\\
    $(2)\implies(3)$: 设$L$是$S\subset K[x]$的分裂域, $\sigma\in\Aut{\overline{K}}$固定$K$不动.
    那么$\sigma$也固定$S$中多项式的系数不动, 从而$S$中多项式的根被排列.
    而$L$由$S$中多项式的根的生成, 所以$L$的元素也被排列, 从而有$\sigma(L)=L$.\\
    $(3)\implies(1)$: 设$p(x)\in K[x]$具有一个根$\alpha\in L$, 不妨设$p(x)$不可约.
    设$\beta\in\overline{K}$是$p(x)$的另一个根, 由于$p(x)$不可约, 所以存在固定$K$不动的同构$K(\alpha)\to K(\beta)$.
    按照同构延拓定理 (定理~\ref{iso ext thm}), 这个同构可以延拓为$\overline{K}$的自同构$\sigma:\overline{K}\to\overline{K}$
    \[\begin{tikzcd}
        K \ar[d, "="] \ar[r] & K(\alpha) \ar[d, "\sim"] \ar[r] & \overline{K} \ar[d, "\sigma"]\\
        K \ar[r] & K(\beta) \ar[r] & \overline{K}
    \end{tikzcd}\]
    那么按照假设, $\sigma$将$L$中的$\alpha$映成$L$中的$\beta$, 即$\beta\in L$.
    从而$p(x)$的根均在$L$中, $p(x)$在$L[x]$中分裂.
\end{proof}

然后我们关心一个多项式的根是不是都是不同的, 与此相关的则是可分扩张的概念.
\begin{defn}
    设$K$是域, $f(x)\in K[x]$, 如果$f(x)$在$\overline{K}$中没有重根, 那么称$f(x)$是一个\textbf{可分多项式}.
    如果$K\subset L$, $\alpha\in L$是一个可分多项式的根, 那么称$\alpha$是\textbf{可分元}.
    如果$L/K$中每个元素都是可分的, 那么称$L/K$是\textbf{可分扩张}.
\end{defn}

\subsection{Galois扩张}

\begin{defn}
    设$K\subset L$是域扩张, 如果$L/K$同时是正规且可分的扩张, 那么称$L/K$为\textbf{Galois扩张}.
\end{defn}

我们将会看到, Galois扩张具有良好的性质.

在讨论Galois扩张之前, 我们需要引入\textit{Galois群}和\textit{不动域}的概念.
在同构延拓定理中我们得到了一类固定底域不动的域同构, 如果这些同构是一个扩域到自身的自同构, 那么将其收集起来可以得到一个群.
那么我们便得到了定义:
\begin{defn}
    设$K\subset L$是域扩张, $L$固定$K$不动的\textbf{Galois群} (当$K$指代明确时, 也称为$L$的Galois群) 定义为所有固定$K$不动的$L$自同构, 记为$\Gal(L/K)$.
\end{defn}

同时我们还会反过来考虑, $\Gal(L/K)$或者它的子群$G$固定的元素可能还有$K$之外的元素.
容易证明$G$固定的元素构成一个域, 我们便得到了定义
\begin{defn}
    设$K\subset L$是域扩张, $\Gal(L/K)$的子群$G$的\textbf{固定域}是$L$中所有在$G$作用下固定不动的元素构成的子域, 记为$L^G$.
\end{defn}

有了以上概念之后, 我们可以来讨论有限Galois扩张的性质.

\begin{prop}\label{property of galois ext 1}
    对有限域扩张$K\subset L$, 以下命题等价:
    \begin{enumerate}[\rm (1)]
        \item $L/K$是Galois扩张, 即$L/K$正规且可分;
        \item $[L:K]=|\Gal(L/K)|$;
        \item $K=L^{\Gal(L/K)}$;
        \item $L$是某个可分多项式的分裂域.
    \end{enumerate}
\end{prop}

在证明命题~\ref{property of galois ext 1}~之前, 我们需要一个引理.

\begin{lem}\label{artin}
    设$L/K$是有限扩张, $X$是域, $\sigma:K\to X$是嵌入.
    那么使得下图交换的嵌入$L\to X$至多有$[L:K]$个.
    \[\begin{tikzcd}
        K \ar[r] \ar[d,"\sigma"] & L \ar[ld, dashed] \\
        X & 
    \end{tikzcd}\]
\end{lem}
\begin{proof}
    当$L/K$是单扩张, 即$L=K(\alpha)$时, $\alpha$具有$[L:K]$次的极小多项式$p$.
    对于任意一个嵌入$f:L\to X$, 有$f(L)=f(K)(f(\alpha))$, 那么对于不同的嵌入, $\alpha$的像均不同.
    但$\alpha$的像均为$f(p)$的根, 由Lagrange定理, 根至多有$[L:K]$个, 所以嵌入至多有$[L:K]$个.
    对一般的$L=K(\alpha_1,\cdots,\alpha_n)$, 考虑扩张链
    \[K\subset K(\alpha_1)\subset\cdots\subset K(\alpha_1,\cdots,\alpha_n)=L\]
    我们可以将$\sigma$提升$n-1$次, 每次从$K(\alpha_1,\cdots,\alpha_{i})$到$K(\alpha_1,\cdots,\alpha_{i+1})$, 每次提升至多有$[K(\alpha_1,\cdots,\alpha_{i+1}):K(\alpha_1,\cdots,\alpha_{i})]$种取法.
    因此至多有
    \[[K(\alpha_1):K]\cdots[K(\alpha_1,\cdots,\alpha_{n}):K(\alpha_1,\cdots,\alpha_{n-1})]=[L:K]\]
    个嵌入.
\end{proof}

\begin{proof}[命题~\ref{property of galois ext 1}~的证明]
    我们按照$(1)\implies(2)\implies(3)\implies(4)\implies(1)$的顺序证明他们等价.\\
    $(1)\implies(2)$: 考虑在引理~\ref{artin}~中取$X=\overline{K}$, 不妨设$L\subset\overline{K}$.
    由于$L$是正规扩张, 所有保持$K$不动的嵌入$L\to\overline{K}$均将$L$映为自身, 从而刚好构成$\Gal(L/K)$.
    而由于$L$是可分扩张, 每个单扩张的极小多项式在$\overline{K}$中恰好有其次数个根, 所以引理~\ref{artin}~中的等号全部成立.
    因此就有$\Gal(L/K)=[L:K]$.\\
    $(2)\implies(3)$: 注意到$K\subset L^{\Gal(L/K)}\subset L$, 而$L$固定$L^{\Gal(L/K)}$不动的自同构有$|\Gal(L/K)|$个, 因此由引理可以得到
    \[[L:K]=|\Gal(L/K)|\leq [L:L^{\Gal(L/K)}]\leq [L:K]\]
    因此$[L:K]=[L:L^{\Gal(L/K)}]$, 所以$K=L^{\Gal(L/K)}$.\\
    $(3)\implies(4)$: 设$L=K(\alpha_1,\cdots,\alpha_n)$, 为每个$\alpha_i$定义多项式
    \[f_i(x)=\prod_{\sigma\in\Gal(L/K)}(x-\sigma(\alpha_i))\]
    由于$\alpha_i\notin K$, 所以每个$\sigma(\alpha_i)$均不相同, $f_i$是可分的; 而$f_i$的系数被$\Gal(L/K)$中的所有元素固定, 在$L^{\Gal(L/K)}=K$中, 所以$f_i(x)\in K[x]$.
    那么取$f=f_1f_2\cdots f_n$, 可知$L$是$f$的分裂域, 且$f$是可分的.\\
    $(4)\implies(1)$: 平凡.
\end{proof}

\section{Galois理论基本定理}

我们在本节陈述并证明Galois理论基本定理.

\begin{thm}[Galois理论基本定理]
    设$K\subset L$是有限Galois扩张, 中间域$K\subset M\subset L$, 子群$1\subset H\subset\Gal(L/K)$.
    那么存在对应
    \[\begin{tikzcd}
        M \ar[r, mapsto] & \Gal(L/M)\\
        L^H & \ar[l, mapsto] H
    \end{tikzcd}\]
    满足$M=L^{\Gal(L/M)},H=\Gal(L/L^H)$.
    特别地, 如果$\Gal(L/M)\lhd\Gal(L/K)$, 那么$M/K$是正规扩张.
\end{thm}

\appendix
\chapter{正文中省略的证明}

\section{Sylow定理}\label{proof of sylow}

\section{代数闭包的存在性}\label{proof of alg closure}

证明代数闭包存在性之前, 我们需要一个引理.
\begin{lem}
    设$L/K$是代数扩张, 那么有$|L|\leq\max\{|K|,|\mathbb{N}|\}$.
\end{lem}
\begin{proof}
    我们有分解
    \[L=\bigcup_{n\geq 1}\{\alpha\in L:\ \deg\alpha=n\}\]
    而对每个$\{\alpha\in L:\ \deg\alpha=n\}$中的元素$\alpha$, $\alpha$与另外至多$n-1$个元素与$K$中$n$个系数决定的首一多项式对应, 从而有
    \[\{\alpha\in L:\ \deg\alpha=n\}\subset [n]\times K^n\]
    对无限的$K$而言, $|[n]\times K^n|=|K|$, 从而
    \begin{align*}
        |L|&=\left|\bigcup_{n\geq 1}\{\alpha\in L:\ \deg\alpha=n\}\right|\\
        &\leq|\mathbb{N}\times K|\\
        &=|K|
    \end{align*}
    对有限的$F$而言, $|[n]\times K^n|=n|K|^n\leq|\mathbb{N}|$, 此时
    \begin{align*}
        |L|&=\left|\bigcup_{n\geq 1}\{\alpha\in L:\ \deg\alpha=n\}\right|\\
        &\leq|\mathbb{N}\times\mathbb{N}|\\
        &=|\mathbb{N}|
    \end{align*}
    综上, 可以得到
    \[|L|\leq\max\{|K|,|\mathbb{N}|\}\qedhere\]
\end{proof}

\begin{proof}[代数闭包存在性的证明]
    设$A$是$K$上所有代数扩域构成的类.
    取$S$满足$F\subset S$且$|S|>\max\{|K|,|\mathbb{N}|\}$, 那么由引理, $K$的代数扩张均包含在$S$中, 从而$A\subset\mathcal{P}(S)$是一个集合.
    使用包含关系作为偏序, 那么注意到对任意一条链$c:(\{K_i\},\subset)$, 易见$\bigcup_{i\geq 1}K_i$是$c$的一个上界.
    因此由Zorn引理, $A$中存在极大元$M$.
    断言在$M$中任意$p(x)\in K[x]$分裂.
    否则假设存在一个$p(x)$在$M$上不能分解为一次因式的乘积, 那么设$p(x)$在$M$上具有分裂域$E$, $E/M,M/K$都是代数扩张, 从而$E/K$是代数扩张 (推论~\ref{alg of alg}), $E\in A$.
    然而$M\subsetneq E$, 这与$M$在$A$中的极大性矛盾.
    因此$M$中任意$p(x)\in K[x]$分裂, 取$M$的由$K[x]$中所有多项式的根生成的子域$\overline{K}$即得到$K$的代数闭包.
    (证明中用到的集合论结论可以参考~\parencite[附录2第2, 3节]{Lang})
\end{proof}

\section{同构延拓定理}\label{proof of iso ext thm}

\begin{proof}[同构延拓定理的证明]
    设$A$是由子域与嵌入$(F,\tau)$构成的集合, 其中$K\subset F\subset E$且使得下图交换
    \[\begin{tikzcd}
        E' & & \\
        K\ar[u, "\sigma"] \ar[r] & F \ar[ul, "\tau"'] \ar[r] & E
    \end{tikzcd}\]
    我们在$A$上定义偏序$(F,\tau)\prec(F',\tau')$当且仅当$F\subset F'$且$\tau'|_F=\tau$.
    对任意一条链$\{(F_i,\tau_i)\}$, 取$F=\bigcup_{i\geq 0}F_i$, $\tau:F\to E'$满足$\tau|_{F_i}=\tau_i$.
    那么容易验证$(F,\tau)$是这条链的一个上界.
    由Zorn引理, $A$中存在一个极大元$(M,\tilde{\sigma})$.
    断言$M=E$. 否则的话存在一个$S$中的多项式$p(x)$在$M$上不分裂, 那么对$p(x)$的一个根$\alpha$, 可以按下图延拓得到$\tilde{\sigma}':M(\alpha)\to E'$
    \[\begin{tikzcd}
         & E'\\
        M(\alpha)\ar[r, "\tilde{\sigma}_\alpha"] \ar[ur, dashed, "\tilde{\sigma}'"] & \tilde{\sigma}(M)(\alpha') \ar[u]\\
        M \ar[r, "\tilde{\sigma}"] \ar[u] & \tilde{\sigma}(M) \ar[u]
    \end{tikzcd}\]
    这与$M$的极大性矛盾, 所以$M=E$.
    注意到$E$包含了$S$中所有多项式的根, 并被$\tilde{\sigma}$一一地映到$E'$中.
    而$E'$是包含$S'$中所有多项式的根的最小的域, 所以一定有$\tilde{\sigma}(E)=E'$.
    因此命题得证.
\end{proof}

\printbibliography[heading=bibintoc]

\end{document}