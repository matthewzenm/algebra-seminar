\chapter{前言}

本讲义是2023年7月作者举办的面向新二年级同学的近世代数讲义.
使用讲义的时候, 作者默认了读者学习了北师大的高等代数I, II课程.

一部分出于懒惰, 一部分美其名曰出于对效率的追求, 本讲义带有很强的``速通''性质.
在这个意义下, 本讲义在尝试如何给出一份\textit{极小}的近世代数入门.
因此读者可以发现, 在正文中我们几乎没有给出例子, 关于各种技术性的细节与结论也给的不多, 我们给出的命题也不一定是最一般的.
为了弥补例子缺乏这个缺陷, 作者讲讨论班时在每次讨论班的结尾会补充一些例子.
这些例子也包括在了每一章的末尾.

本讲义采用的讲法与标准的教科书不完全相同.
首先我们以一种范畴化的视角给出最初的定义, 即我们定义群与环之后立刻给出同态的定义.
通过同态, 我们可以给出子结构的定义以及正规子群与理想的定义.
接下来我们考虑群和环共通的结构: 商结构, 乘积结构与生成结构, 这些结构给出了代数学最基本的观念.
在考虑完群和环共通的结构之后, 我们分别考虑群和环进一步的性质:
群则讨论群作用, 单群, Sylow定理与低阶有限群的分类;
而环则讨论交换环的素理想, 极大理想, 以及三类具有分解性质的整环.
结束群与环的讨论之后, 我们进入域的部分.
我们分别用两章介绍域的扩张与Galois理论, 我们仍然在这里采用极小的讲法, 为此, 我们直接引入域上任意一族多项式的分裂域, 通过这个得到代数闭域与正规扩张的性质.

本讲义的内容远非作者原创, 我们也参考了许多书籍及课程等等.
部分参考书籍与我们推荐阅读的书籍如下:
对于中文书籍, \parencite{BNU}~是北师大近世代数课程的教材; 而~\parencite{wwli}~则是一本比较``升级''的教材, 介绍了更加现代的内容.
对于英文书籍, \parencite{Hungerford}~是标准的教材;
\parencite{Chapter0}~是一本偏向入门的书籍, 但在一开始便以较高观点引入范畴等内容, 适合作为研究生级别教材学习;
\parencite{Lang}~是著名的字典, 以大而全闻名, 适合用来查阅.

作者认为前言应当是在一本书的创作结束时撰写的.
现在讲义的编写接近尾声, 作者感到了一阵阵空虚---因为作者不知道这份讲义还会在什么场景下被人使用.
也许这份讲义编写出来就是作者的自娱自乐罢了.
因此, 只要有读者愿意读作者的这份讲义, 作者就会很开心了.
如果进一步真的有人能够通过这份讲义速成或复习近世代数, 那么就更好了.

\begin{flushright}
    \textit{魔法少女Alkali}\\
    \today
\end{flushright}

\chapter{预备知识}

我们具体列举希望读者掌握的预备知识如下.

首先, 讲义中会使用与北师大高等代数课程不同的记号$\mathbb{Z}/n\mathbb{Z}$表示模$n$剩余类环, 并且会直接使用同余记号$\mathrm{mod}$记属于同一等价类的元素.

接下来我们定义群环域.
\begin{defn}设$X$是一个集合, 具有二元运算$*:X\times X\to X$, 并有公理
    \begin{itemize}
        \item[G1] 对$a,b,c\in X$有$(a*b)*c=a*(b*c)$;
        \item[G2] 存在$e\in X$, 使得对任意$a\in X$有$a*e=e*a=a$;
        \item[G3] 对$a\in X$, 存在$b\in X$使得$a*b=b*a=e$;
        \item[Ab] 对任意$a,b\in X$, 有$a*b=b*a$;
    \end{itemize}
    如果$X$上还有另一二元运算$\cdot:X\times X\to X$, 此时还有公理
    \begin{itemize}
        \item[R1] 对$a,b,c\in X$有$(a\cdot b)\cdot c=a\cdot(b\cdot c)$;
        \item[R2] 存在$1\in X$, 使得对任意$a\in X$有$a\cdot 1=1\cdot a=a$;
        \item[R3] 对任意$a,b\in X$, 有$a\cdot b=b\cdot a$;
        \item[Ds] 对$a,b,c\in X$有$(a*b)\cdot c=a\cdot c*b\cdot c, a\cdot(b*c)=a\cdot b*a\cdot c$;
        \item[F] 对任意$a\in X\backslash\{e\}$, 存在$b\in X$使得$a\cdot b=1$;
    \end{itemize}
    如果$X$满足G1$\sim$G3, 那么称$X$是一个\textbf{群}; 如果群$X$还满足Ab, 则称$X$是一个\textbf{Abel群}.
    如果$X$是Abel群, 且满足R1与Ds, 那么称$X$是一个\textbf{环}; 如果环$X$满足R2, 那么称$X$\textbf{含幺}; 如果环$X$满足R3, 那么称$X$是\textbf{交换环}.
    如果$X$是交换环且满足F, 那么称$X$是一个\textbf{域}.
\end{defn}

\begin{sym}
    习惯上, 一般对群的运算会采用两种记号: 一种是乘法记号$\cdot$, 在实际书写中会直接省略这个点; 另一种是加法记号$+$.
    乘法记号会用在一般的群或者环满足R1与Ds的运算上, 加法记号会用在Abel群的运算上.
    运用乘法记号时, G3中定义的逆元会记作$a^{-1}$.
    运用加法记号时, G2中定义的加法零元记为$0$, G3中定义的逆元记为$-a$.
    对域而言, F中定义的逆元记为$a^{-1}$.

    本讲义中如果不另外说明, 环都是含幺的.

    习惯上会用一些特定的字母表示特定的代数结构, 例如群用$G$表示, 环用$R$表示, 交换环用$A$表示, 域用$F$或$k$表示.
\end{sym}

然后是置换群的基本概念.
\begin{prop}
    集合$\{1,2,\cdots,n\}$到自身的双射构成群, 记为$S_n$, 称为{\bf $n$阶置换群}.
\end{prop}

\begin{sym}
    对$\sigma\in S_n$, 我们会用
    \[\begin{pmatrix}
        1 & \cdots & n\\
        \sigma(1) & \cdots & \sigma(n)
    \end{pmatrix}\]
    来表示一个置换.
\end{sym}

\begin{prop}
    一个{\bf 轮换}定义为
    \[\begin{pmatrix}
        a_1 & a_2 & \cdots & a_n\\
        a_2 & a_3 & \cdots & a_1
    \end{pmatrix}\]
    记为$(a_1a_2\cdots a_n)$.
    每一个置换都可以写成不相交轮换的乘积, 这种写法在不计次序的意义下唯一.
\end{prop}

最后是关于矩阵的两种群.
\begin{defn}
    域$k$上的$n$阶\textbf{一般线性群}$\GL_n(k)$定义为$k$上所有可逆的$n$阶矩阵构成的群.
    域$k$上的$n$阶\textbf{特殊线性群}$\SL_n(k)$定义为$k$上所有行列式为$1$的$n$阶矩阵构成的群.
\end{defn}