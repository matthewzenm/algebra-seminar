\chapter{正文中省略的证明}

\section{Sylow定理}\label{proof of sylow}

\section{代数闭包的存在性}\label{proof of alg closure}

证明代数闭包存在性之前, 我们需要一个引理.
\begin{lem}
    设$L/K$是代数扩张, 那么有$|L|\leq\max\{|K|,|\mathbb{N}|\}$.
\end{lem}
\begin{proof}
    我们有分解
    \[L=\bigcup_{n\geq 1}\{\alpha\in L:\ \deg\alpha=n\}\]
    而对每个$\{\alpha\in L:\ \deg\alpha=n\}$中的元素$\alpha$, $\alpha$与另外至多$n-1$个元素与$K$中$n$个系数决定的首一多项式对应, 从而有
    \[\{\alpha\in L:\ \deg\alpha=n\}\subset [n]\times K^n\]
    对无限的$K$而言, $|[n]\times K^n|=|K|$, 从而
    \begin{align*}
        |L|&=\left|\bigcup_{n\geq 1}\{\alpha\in L:\ \deg\alpha=n\}\right|\\
        &\leq|\mathbb{N}\times K|\\
        &=|K|
    \end{align*}
    对有限的$F$而言, $|[n]\times K^n|=n|K|^n\leq|\mathbb{N}|$, 此时
    \begin{align*}
        |L|&=\left|\bigcup_{n\geq 1}\{\alpha\in L:\ \deg\alpha=n\}\right|\\
        &\leq|\mathbb{N}\times\mathbb{N}|\\
        &=|\mathbb{N}|
    \end{align*}
    综上, 可以得到
    \[|L|\leq\max\{|K|,|\mathbb{N}|\}\qedhere\]
\end{proof}

\begin{proof}[代数闭包存在性的证明]
    设$A$是$K$上所有代数扩域构成的类.
    取$S$满足$F\subset S$且$|S|>\max\{|K|,|\mathbb{N}|\}$, 那么由引理, $K$的代数扩张均包含在$S$中, 从而$A\subset\mathcal{P}(S)$是一个集合.
    使用包含关系作为偏序, 那么注意到对任意一条链$c:(\{K_i\},\subset)$, 易见$\bigcup_{i\geq 1}K_i$是$c$的一个上界.
    因此由Zorn引理, $A$中存在极大元$M$.
    断言在$M$中任意$p(x)\in K[x]$分裂.
    否则假设存在一个$p(x)$在$M$上不能分解为一次因式的乘积, 那么设$p(x)$在$M$上具有分裂域$E$, $E/M,M/K$都是代数扩张, 从而$E/K$是代数扩张 (推论~\ref{alg of alg}), $E\in A$.
    然而$M\subsetneq E$, 这与$M$在$A$中的极大性矛盾.
    因此$M$中任意$p(x)\in K[x]$分裂, 取$M$的由$K[x]$中所有多项式的根生成的子域$\overline{K}$即得到$K$的代数闭包.
    (证明中用到的集合论结论可以参考~\parencite[附录2第2, 3节]{Lang})
\end{proof}

\section{同构延拓定理}\label{proof of iso ext thm}

\begin{proof}[同构延拓定理的证明]
    设$A$是由子域与嵌入$(F,\tau)$构成的集合, 其中$K\subset F\subset E$且使得下图交换
    \[\begin{tikzcd}
        E' & & \\
        K\ar[u, "\sigma"] \ar[r] & F \ar[ul, "\tau"'] \ar[r] & E
    \end{tikzcd}\]
    我们在$A$上定义偏序$(F,\tau)\prec(F',\tau')$当且仅当$F\subset F'$且$\tau'|_F=\tau$.
    对任意一条链$\{(F_i,\tau_i)\}$, 取$F=\bigcup_{i\geq 0}F_i$, $\tau:F\to E'$满足$\tau|_{F_i}=\tau_i$.
    那么容易验证$(F,\tau)$是这条链的一个上界.
    由Zorn引理, $A$中存在一个极大元$(M,\tilde{\sigma})$.
    断言$M=E$. 否则的话存在一个$S$中的多项式$p(x)$在$M$上不分裂, 那么对$p(x)$的一个根$\alpha$, 可以按下图延拓得到$\tilde{\sigma}':M(\alpha)\to E'$
    \[\begin{tikzcd}
         & E'\\
        M(\alpha)\ar[r, "\tilde{\sigma}_\alpha"] \ar[ur, dashed, "\tilde{\sigma}'"] & \tilde{\sigma}(M)(\alpha') \ar[u]\\
        M \ar[r, "\tilde{\sigma}"] \ar[u] & \tilde{\sigma}(M) \ar[u]
    \end{tikzcd}\]
    这与$M$的极大性矛盾, 所以$M=E$.
    注意到$E$包含了$S$中所有多项式的根, 并被$\tilde{\sigma}$一一地映到$E'$中.
    而$E'$是包含$S'$中所有多项式的根的最小的域, 所以一定有$\tilde{\sigma}(E)=E'$.
    因此命题得证.
\end{proof}